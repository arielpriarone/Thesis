% for glossary entry
% @entry{bird,
%     name={bird},
%     description = {feathered animal},
%     see={[see also]{duck,goose}}
% }

% if this bib file does not work, try using \input{file.tex}
% where all the \newabbreviation commands have been inserted
% containing all the definitions

% Gls to capitalize first letter
% GLS for full uppercase
% for abbreviations also
% glsxtrshort for abbreviation
% similar for long, full, and capital configurations, add pl at the end for plurals
% glsentryshort, long, plural (referred to shorts) must be used when in section titles
% glslink to allow the link but use a different text (as for href)


% if you want to use also description for the abbreviations/acronyms, you should use bib2gls and define all the entries in a bib file, which is incompatible with Overleaf
\newacronym{lcm}{LCM}{Least Common Multiple}
\newacronym{pm}{PM}{Predective maintenance}
\newacronym{AI}{AI}{Artificial intelligence}
\newacronym{pdm}{PdM}{Predictive Maintenance}
\newacronym{rm}{RM}{Reactive Maintenance}
\newacronym{aka}{a.k.a.}{Also Known As}
\newacronym{pof}{POF}{Pareto Optimal Front}

%Symbols
\glsxtrnewsymbol[
    description={\textbf{Cluster} A set of objects that are more similar to each other than to those in other clusters.}]
    {sym:cluster}
    {\ensuremath{\vect{\mathcal{C}}}}

\glsxtrnewsymbol[
    description={\textbf{Snapshot} A set of features that describe the state of a system at a given time.}]
    {sym:snap}
    {\ensuremath{\vect{\mathcal{S}}}}
    
\glsxtrnewsymbol[
    description={\textbf{Snapshots Set} A set of snapshots \gls{sym:snap}.}]
    {sym:snapset}
    {\ensuremath{\vect{\mathbf{S}}}}

\glsxtrnewsymbol[
    description={\textbf{Distance} Vector difference between two spoints in the features space.}]
    {sym:dist}
    {\ensuremath{\vect{d}}}

\glsxtrnewsymbol[
    description={\textbf{Radius} Euclidean distance between the centroid \gls{sym:cent} of a cluster and its farthest point.}]
    {sym:radius}
    {\ensuremath{\vect{r}}}

\glsxtrnewsymbol[
    description={\textbf{Centroid} Point in the features space that represents a cluster. Ideally it is the center of mass of the cluster it represents.}]  
    {sym:cent}
    {\ensuremath{\vect{c}}}


%Glossary
\newglossaryentry{glo:std}{
    name={standardized},
    description={a signal that has been transformed to have a zero mean and unit variance}}

\newglossaryentry{glo:heuristic}{
    name={heuristic},
    description={\quoted{any device, be it a 
    program, rule, piece of knowledge, etc., which one is not 
    entirely confident will be useful in providing a practical 
    solution, but which one has reason to believe will be useful, and 
    which is added to a problem-solving system in expectation that 
    on average the performance will improve.}\cite{romanycia1985heuristic}}}




