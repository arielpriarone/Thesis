% for glossary entry
% @entry{bird,
%     name={bird},
%     description = {feathered animal},
%     see={[see also]{duck,goose}}
% }

% if this bib file does not work, try using \input{file.tex}
% where all the \newabbreviation commands have been inserted
% containing all the definitions

% Gls to capitalize first letter
% GLS for full uppercase
% for abbreviations also
% glsxtrshort for abbreviation
% similar for long, full, and capital configurations, add pl at the end for plurals
% glsentryshort, long, plural (referred to shorts) must be used when in section titles
% glslink to allow the link but use a different text (as for href)


% if you want to use also description for the abbreviations/acronyms, you should use bib2gls and define all the entries in a bib file, which is incompatible with Overleaf
\newacronym{lcm}{LCM}{Least Common Multiple}
\newacronym{AI}{AI}{Artificial intelligence}
\newacronym{aka}{a.k.a.}{Also Known As}
\newacronym{wrt}{w.r.t.}{With Respect To}
\newacronym{pof}{POF}{Pareto Optimal Front}

\newacronym{pm}{PM}{Proactive Maintenance} %% maintenance techniques
\newacronym{pvm}{PvM}{Preventive Maintenance}
\newacronym{rm}{RM}{Reactive Maintenance}
\newacronym{pdm}{PdM}{Predictive Maintenance}
\newacronym{cbm}{CM}{Condition Based Maintenance}
\newacronym{nd}{ND}{Novelty Detection}
\newacronym{fd}{FD}{Fault Detection}    
\newacronym{ml}{ML}{Machine Learining}  
\newacronym{uml}{UML}{Unsupervised Machine Learining}
\newacronym{pc}{PC}{Personal Computer}

%Symbols
\glsxtrnewsymbol[
    description={\textbf{Cluster} A set of objects that are more similar to each other than to those in other clusters.}]
    {sym:cluster}
    {\ensuremath{\vect{\mathcal{C}}}}

\glsxtrnewsymbol[
    description={\textbf{Snapshot} A set of features that describe the state of a system at a given time.}]
    {sym:snap}
    {\ensuremath{\vect{\mathcal{S}}}}
    
\glsxtrnewsymbol[
    description={\textbf{Snapshots Set} A set of snapshots \gls{sym:snap}.}]
    {sym:snapset}
    {\ensuremath{\vect{\mathbf{S}}}}

\glsxtrnewsymbol[
    description={\textbf{Distance} Vector difference between two spoints in the features space.}]
    {sym:dist}
    {\ensuremath{\vect{d}}}

\glsxtrnewsymbol[
    description={\textbf{Radius} Euclidean distance between the centroid \gls{sym:cent} of a cluster and its farthest point.}]
    {sym:radius}
    {\ensuremath{\vect{r}}}

\glsxtrnewsymbol[
    description={\textbf{Centroid} Point in the features space that represents a cluster. Ideally it is the center of mass of the cluster it represents.}]  
    {sym:cent}
    {\ensuremath{\vect{c}}}


%Glossary
\newglossaryentry{glo:std}{
    name={standardized},
    description={a signal that has been transformed to have a zero mean and unit variance}}

\newglossaryentry{glo:snap}{
    name={snapshot},
    description={an array of features that describe the state of a system in a specific time period. It's filled with any metric (time domanin, frequency domain, etc.)}}

\newglossaryentry{glo:heuristic}{
    name={heuristic},
    description={\quoted{any device, be it a 
    program, rule, piece of knowledge, etc., which one is not 
    entirely confident will be useful in providing a practical 
    solution, but which one has reason to believe will be useful, and 
    which is added to a problem-solving system in expectation that 
    on average the performance will improve.}\cite{romanycia1985heuristic}}}

\newglossaryentry{glo:cent}{
    name={centroid},
    description={The center of a cluster. It's the point that minimize the sum of the distances between itself and all the points in the cluster. From a phisical point of view, it's the center of mass of the cluster, if all the points of the cluster are treated as equal point masses.}}

\newglossaryentry{glo:edge}{
    name={edge computing},
    description={\quoted{Edge computing is an emerging computing paradigm which refers to a range of networks and devices at or near the user. Edge is about processing data closer to where it's being generated, enabling processing at greater speeds and volumes, leading to greater action-led results in real time.}\cite{edge_computing_accenture}}}

\newglossaryentry{glo:clust}{
    name={cluster},
    description={In a set of data points, a cluster is a subset of the former that are more similar to each other than to the rest of the data points. This is a broad definition that leave to the alrithm applied to perform the clustering the freedom to define what \quoted{similar} means.}}
