% for glossary entry
% @entry{bird,
%     name={bird},
%     description = {feathered animal},
%     see={[see also]{duck,goose}}
% }

% if this bib file does not work, try using \input{file.tex}
% where all the \newabbreviation commands have been inserted
% containing all the definitions

% Gls to capitalize first letter
% GLS for full uppercase
% for abbreviations also
% glsxtrshort for abbreviation
% similar for long, full, and capital configurations, add pl at the end for plurals
% glsentryshort, long, plural (referred to shorts) must be used when in section titles
% glslink to allow the link but use a different text (as for href)


% if you want to use also description for the abbreviations/acronyms, you should use bib2gls and define all the entries in a bib file, which is incompatible with Overleaf

\newacronym{pm}{PM}{Proactive Maintenance} %% maintenance techniques
\newacronym{pvm}{PvM}{Preventive Maintenance}
\newacronym{rm}{RM}{Reactive Maintenance}
\newacronym{pdm}{PdM}{Predictive Maintenance}
\newacronym{cbm}{CBM}{Condition Based Maintenance}
\newacronym{iot}{IOT}{Internet Of Things}
\newacronym{os}{OS}{Operating System}
\newacronym{rul}{RUL}{\gls{glo:rul}}


\newacronym{nd}{ND}{Novelty Detection}  %% machine learning
\newacronym{fd}{FD}{Fault Detection}    
\newacronym{ml}{ML}{Machine Learining}  
\newacronym{uml}{UML}{Unsupervised Machine Learining}
\newacronym{svm}{SVM}{Support Vector Machine}
\newacronym{nu_svm}{$\nu$-SVM}{One Class \gls{svm}}
\newacronym{iforest}{iForest}{Isolation Forest}
\newacronym{mlp}{MLP}{Multilayered Perceptront}
\newacronym{dl}{DL}{\gls{glo:deep}}
\newacronym{ann}{ANN}{Artificial Neural Network}
\newacronym{dt}{DT}{Decision Tree}
\newacronym{knn}{$k$-NN}{k-Nearest Neighbors}
\newacronym{pf}{PF}{Particle Filter}
\newacronym{art}{ART}{Adaptive Resonance Theory}
\newacronym{som}{SOM}{Self-Organizing Map}
\newacronym{ae}{AE}{Autoencoder}
\newacronym{cnn}{CNN}{Convolutional Neural Network}
\newacronym{rnn}{RNN}{Recurrent Neural Network}
\newacronym{dbn}{DBN}{Deep Belief Network}
\newacronym{gan}{GAN}{Generative Adversarial Network}
\newacronym{tl}{TL}{Transfer Learning}
\newacronym{dlr}{DLR}{Deep Reinforcement Learning}
\newacronym{cep}{CEP}{Cepstral Editing Procedure}
\newacronym{pw}{PW}{Pre-Whitening }
\newacronym{lr}{LR}{Linear Regressor}
\newacronym{rf}{RF}{Random Forest}
\newacronym{ls}{LS}{Least Squares}
\newacronym{gd}{GD}{Gradient Descent}
\newacronym{sgd}{SGD}{Stochastic Gradient Descent}
\newacronym{cart}{CART}{Classification and Regression Tree}
\newacronym{mla}{MLA}{\gls{glo:mla}}
\newacronym{fa}{FA}{\gls{glo:featAg}}
\newacronym{fieldAg}{FiA}{\gls{glo:fieldagent}}
\newacronym{tn}{TN}{Transfer Learning}
\newacronym{cft}{CFT}{Continuous Fourier Transform}

\newacronym{pc}{PC}{Personal Computer} %% miscellanea
\newacronym{lcm}{LCM}{Least Common Multiple}
\newacronym{AI}{AI}{Artificial intelligence}
\newacronym{aka}{a.k.a.}{Also Known As}
\newacronym{wrt}{w.r.t.}{With Respect To}
\newacronym{pof}{POF}{Pareto Optimal Front}
\newacronym{dbscan}{DBSCAN}{Density-Based Spatial Clustering of Applications with Noise}
\newacronym{gmm}{GMM}{Gaussian Mixture Model}
\newacronym{bgmm}{BGMM}{Bayesan Gaussian Mixture Model}
\newacronym{em}{EM}{Expetaion Maximization}
\newacronym{pdf}{PDF}{Probability Density Function}
\newacronym{aic}{AIC}{Akaike Information Criterion}
\newacronym{bic}{BIC}{Bayesian Information Criterion}
\newacronym{lof}{LOF}{Local Outlier Factor}
\newacronym{rtd}{RTD}{resistance temperature detector}
\newacronym{lcsr}{LCSR}{Loop CurrentStep Response}
\newacronym{bpfo}{BPFO}{Ballpass frequency, outer race}
\newacronym{bpfi}{BPFI}{Ballpass frequency, inner race}
\newacronym{ftf}{FTF}{Fundamental train frequency (cage speed)}
\newacronym{bsf}{BSF}{Ball (roller) spin frequency}
\newacronym{mse}{MSE}{Mean Squared Error}
\newacronym{rmse}{RMSE}{Root Mean Squared Error}
\newacronym{rms}{RMS}{Root Mean Squared}
\newacronym{nasa}{NASA}{National Aeronautics and Space Administration (\gls{usa})}
\newacronym{ims}{IMS}{Center for Intelligent Maintenance Systems}
\newacronym{usa}{USA}{United States of America}
\newacronym{iso}{ISO}{International Organization for Standardization}
\newacronym{dft}{DFT}{Discrete Fourier Transform}
\newacronym{fft}{FFT}{Fast Fourier Transform}
\newacronym{wpd}{WPD}{Wavelet Packet Decomposition}
\newacronym{json}{JSON}{{\gls{glo:json}}}
\newacronym{sql}{SQL}{{\gls{glo:sql}}}
\newacronym{bson}{BSON}{Binary \gls{glo:json}}
\newacronym{cnc}{CNC}{Computer Numerical Control}
\newacronym{ie}{i.e.}{\quoted{id est} (that is)}
\newacronym{adc}{ADC}{Analog to Digital Converter}
\newacronym{csv}{CSV}{Comma Separated Values}
\newacronym{cli}{CLI}{Command Line Interface}
\newacronym{hal}{HAL}{Hardware Abstraction Library}
\newacronym{ide}{IDE}{Integrated Development Environment}
\newacronym{dma}{DMA}{Direct Memory Access}
\newacronym{gpio}{GPIO}{General Purpose Input Output}
\newacronym{anova}{ANOVA}{Analysis of Variance}
\newacronym{gui}{GUI}{Graphical User Interface}
\newacronym{wt}{WT}{Wavelet Transform}



%Symbols
\glsxtrnewsymbol[
    description={\textbf{Cluster} A set of objects that are more similar to each other than to those in other clusters.}]
    {sym:cluster}
    {\ensuremath{\vect{\mathcal{C}}}}

\glsxtrnewsymbol[
    description={\textbf{Snapshot} A set of features that describe the state of a system at a given time.}]
    {sym:snap}
    {\ensuremath{\vect{\mathcal{S}}}}
    
\glsxtrnewsymbol[
    description={\textbf{Snapshots Set} A set of snapshots \gls{sym:snap}.}]
    {sym:snapset}
    {\ensuremath{\vect{\mathbf{S}}}}

\glsxtrnewsymbol[
    description={\textbf{Distance} Vector difference between two points in the features space.}]
    {sym:dist}
    {\ensuremath{\vect{d}}}

\glsxtrnewsymbol[
    description={\textbf{Radius} Euclidean distance between the centroid \gls{sym:cent} of a cluster and its farthest point.}]
    {sym:radius}
    {\ensuremath{\vect{r}}}

\glsxtrnewsymbol[
    description={\textbf{Centroid} Point in the features space that represents a cluster. Ideally it is the center of mass of the cluster it represents.}]  
    {sym:cent}
    {\ensuremath{\vect{c}}}

\glsxtrnewsymbol[
    description={\textbf{Features} A set of metrics that describe the state of a system.}]
    {sym:feat}
    {\ensuremath{\vect{f}}}

\glsxtrnewsymbol[
    description={\textbf{Feature number} The number of features that describe the state of a system.}]
    {sym:feats}
    {\ensuremath{F}}

\glsxtrnewsymbol[
    description={\textbf{Expected value} the arithmetic mean of the possible values a random variable can take, weighted by the probability of those outcomes}]
    {sym:E}
    {\ensuremath{\E}}




%Glossary
\newglossaryentry{glo:std}{
    name={standardized},
    description={a signal that has been transformed to have a zero mean and unit variance}}

\newglossaryentry{glo:snap}{
    name={snapshot},
    description={an array of features that describe the state of a system in a specific time period. It's filled with any metric (time domain, frequency domain, etc.)}}

\newglossaryentry{glo:heuristic}{
    name={heuristic},
    description={\quoted{any device, be it a 
    program, rule, piece of knowledge, etc., which one is not 
    entirely confident will be useful in providing a practical 
    solution, but which one has reason to believe will be useful, and 
    which is added to a problem-solving system in expectation that 
    on average the performance will improve.}\cite{romanycia1985heuristic}}}

\newglossaryentry{glo:python}{
    name={python},
    description={\quoted{an interpreted, object-oriented, high-level programming language with dynamic semantics. Its high-level built in data structures, combined with dynamic typing and dynamic binding, make it very attractive for Rapid Application Development, as well as for use as a scripting or glue language to connect existing components together.} \cite{python}}}

\newglossaryentry{glo:cent}{
    name={centroid},
    description={The center of a cluster. It's the point that minimizes the sum of the distances between itself and all the points in the cluster. From a physical point of view, it's the center of mass of the cluster, if all the points of the cluster are treated as equal point masses.}}

\newglossaryentry{glo:edge}{
    name={edge computing},
    description={\quoted{Edge computing is an emerging computing paradigm which refers to a range of networks and devices at or near the user. Edge is about processing data closer to where it's being generated, enabling processing at greater speeds and volumes, leading to greater action-led results in real time.}\cite{edge_computing_accenture}}}

\newglossaryentry{glo:clust}{
    name={cluster},
    description={In a set of data points, a cluster is a subset of the former that are more similar to each other than to the rest of the data points. This is a broad definition that leaves to the algorithm applied to perform the clustering the freedom to define what \quoted{similar} means.}}

\newglossaryentry{glo:lin-sep}{
    name={linearly separable},
    description={\quoted{Two sets of data points in a two-dimensional space are said to be linearly separable when they can be completely separable by a single straight line. In general, two groups of data points are separable in a n-dimensional space if they can be separated by an (n-1)-dimensional hyperplane} \cite{lazar2009linearly}.}}

\newglossaryentry{glo:likelihood}{
    name={likelihood function},
    description={The likelihood function quantifies the probability that the observed data would be generated by a specific parametric model.}}

\newglossaryentry{glo:preventivemaintenance}{
    name={Preventive Maintenance},
    description={\quoted{Maintenance carried out at predetermined intervals or according to prescribed criteria. Intended to reduce the probability of failure or the degradation of the functioning of an item} \cite{EN13306:2018}.}
}

\newglossaryentry{glo:predeterminedmaintenance}{
    name={Predetermined Maintenance},
    description={Preventive maintenance carried out at established intervals of time or number of units of use, without previous condition investigation. \emph{Note: Intervals of times or number of unit of use may be established from knowledge of the failure mechanisms of the item\cite{EN13306:2018}.}}
}

\newglossaryentry{glo:conditionbasedmaintenance}{
    name={Condition Based Maintenance},
    description={\quoted{Preventive maintenance including a combination of condition monitoring, inspection, testing, analysis, and ensuing maintenance actions. \emph{Note: Condition monitoring and/or inspection and/or testing may be scheduled, on request, or continuous.}} \cite{EN13306:2018}. }
}

\newglossaryentry{glo:predictivemaintenance}{
    name={Predictive Maintenance},
    description={\quoted{Condition-based maintenance carried out following a forecast derived from repeated analysis or known characteristics and evaluation of the significant parameters of the degradation of the item} \cite{EN13306:2018}.}
}

\newglossaryentry{glo:correctivemaintenance}{
    name={Corrective Maintenance},
    description={\quoted{Maintenance carried out after fault recognition and intended to put an item into a state in which it can perform a required function} \cite{EN13306:2018}.}
}

\newglossaryentry{glo:deferredcorrectivemaintenance}{
    name={Deferred Corrective Maintenance},
    description={Corrective maintenance not immediately carried out after fault detection but is delayed in accordance with given rules \cite{EN13306:2018}.}
}

\newglossaryentry{glo:immediatecorrectivemaintenance}{
    name={Immediate Corrective Maintenance},
    description={Corrective maintenance carried out without delay after a fault has been detected to avoid unacceptable consequences \cite{EN13306:2018}.}
}

\newglossaryentry{glo:scheduledmaintenance}{
    name={Scheduled Maintenance},
    description={Maintenance carried out in accordance with an established time schedule or established number of units of use. \emph{Note: Corrective deferred maintenance may also be scheduled \cite{EN13306:2018}.} }
}

\newglossaryentry{glo:remotemaintenance}{
    name={Remote Maintenance},
    description={Maintenance of an item carried out without physical access by personnel to the item \cite{EN13306:2018}.}
}

\newglossaryentry{glo:onlinemaintenance}{
    name={On Line Maintenance},
    description={\quoted{Maintenance carried out on the item while it is operating and without impact on its performance. \emph{Note: In this type of maintenance, it is important that all safety procedures are followed}}\cite{EN13306:2018}. }
}

\newglossaryentry{glo:onistemaintenance}{
    name={On Site Maintenance},
    description={\quoted{Maintenance carried out at the location where the item is normally located} \cite{EN13306:2018}.}
}

\newglossaryentry{glo:operatormaintenance}{
    name={Operator Maintenance},
    description={Maintenance actions carried out by an operator. \emph{Note: Such maintenance actions should be clearly defined.} \cite{EN13306:2018}}
}

\newglossaryentry{glo:maintenancelevel}{
    name={Maintenance Level},
    description={Level of maintenance. M \cite{EN13306:2018}.aintenance task categorization by complexity \cite{EN13306:2018}.}
}

\newglossaryentry{glo:deep}{
    name={Deep Learning},
    description={\quoted{a class of algorithms which are based on artificial neural networks optimized to work with unstructured data such as images, voice, videos and text} \cite{KOTU2019307}.}
}

\newglossaryentry{glo:trad_ml}{
    name={Traditional \gls{ml}},
    description={algorithm learning from data to perform a task without being explicitly programmed, excluding deep learning algorithms.}
}
    
\newglossaryentry{glo:rul}{
    name={Remaining Useful Life},
    description={The remaining time before system health falls below a defined failure threshold \cite{ISO13381_1}.}
}

\newglossaryentry{glo:hyperparameter}{
    name={hyperparameter},
    description={\quoted{an hyperparameter is a parameter of a learning algorithm (not of the
    model). As such, it is not affected by the learning algorithm itself; it must be set prior
    to training and remains constant during training} \citepage{hands-on-geron2022}{30}.}
}

\newglossaryentry{glo:feature}{
    name={feature},
    description={a feature is an individual measurable property or characteristic of a phenomenon being observed.}
}

\newglossaryentry{glo:polling}{
    name={polling},
    description={to cyclically collect data from a device.}
}

\newglossaryentry{glo:json}{
    name={JavaScript Object Notation},
    description={\quoted{is an open standard file format, and data interchange format, that uses human-readable text to store and transmit data objects consisting of attribute-value pairs and array data types} \cite{json_definition}.}
}

\newglossaryentry{glo:nosql}{
    name={NoSQL},
    description={\quoted{ an approach to database management that can accommodate a wide variety of data models, including key-value, document, columnar and graph formats. A NoSQL database generally means that it is non-relational, distributed, flexible and scalable} \cite{noSQL_definition}}.
}

\newglossaryentry{glo:sql}{
    name={Structured Query Language},
    description={a domain-specific language for managing and querying relational databases, utilizing tables to organize and structure data, facilitating efficient data retrieval, insertion, and modification operations.}
}

\newglossaryentry{glo:commissioning}{
    name={commissioning},
    description={to put a new system into working conditions.}
}

\newglossaryentry{glo:frmwrk}{
    name={framework},
    description={\quoted{a real or conceptual structure intended to serve as a support or guide for the building of something that expands the structure into something useful} \cite{framework_definition}. In software development, a framework is a set of functions and classes and practices that can be used to easily solve a specific problem.}
}

\newglossaryentry{glo:fieldagent}{
    name={Field Agent},
    description={a software agent that polls the data from the field and stores them in a database in a suitable formatted way.}
}

\newglossaryentry{glo:featAg}{
    name={Feature Agent},
    description={a software agent that extracts the features from the data polled by the \gls{glo:fieldagent} and stores them in a database in a suitable formatted way.}
}

\newglossaryentry{glo:mla}{
    name={Machine Learning Agent},
    description={software agent that trains the models, evaluate the metrics on new data, and makes predictions about the future evolution of the metrics. It also interface itself with the operator.}
}

\newglossaryentry{glo:mongodb}{
    name={MongoDB},
    description={a source-available, cross-platform, document-oriented database program. }
}

\newglossaryentry{glo:pickle}{
    name={Pickle format},
    description={\quoted{The pickle module implements binary protocols for serializing and de-serializing a Python object structure. “Pickling” is the process whereby a Python object hierarchy is converted into a byte stream, and “unpickling” is the inverse operation, whereby a byte stream (from a binary file or bytes-like object) is converted back into an object hierarchy}\cite{python_docs}.}
}

\newglossaryentry{glo:agent}{
    name={agent},
    description={\quoted{Software agent, a computer program that performs various actions continuously and autonomously on behalf of an individual or an organization. For example, a software agent may archive various computer files or retrieve electronic messages on a regular schedule. Such simple tasks barely begin to tap the potential uses of software agents, however.} \cite{software_agent}}
}

\newglossaryentry{glo:leadtime}{
    name={Lead Time},
    description={The interval between the detection of a novelty, and a malfunction happening.}
}

\newglossaryentry{glo:sklearn}{
    name={\texttt{sklearn}},
    description={Scikit-learn is a free a Python-based machine learning library. It is commonly used for classification, regression, and clustering tasks. It implements algorithms such as support vector machines, random forests, gradient boosting, k-means, and DBSCAN. The library integrates with most of Python's numerical libraries.}
}