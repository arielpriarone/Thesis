\gls{glo:predictivemaintenance} and Novelty Detection are important topics in modern industrial engineering, aimed at proactively identifying equipment failures before they affect system functionality. Embracing these practices is crucial for reducing equipment downtime and optimizing maintenance efforts. \gls{glo:predictivemaintenance} aims to quantify and forecast the state of degradation of a system. A quite novel frontier is the direct implementation of \gls{glo:predictivemaintenance} within the maintained device, using the principles of Edge Computing.

The fourth industrial revolution is characterized by the integration of Artificial Intelligence and the Internet of Things paradigm into factories. Nowadays, more than a decade since the beginning of this industrial revolution, the maintenance approach remained unchanged in most industrial applications. The primary factor impeding the advancement of the maintenance approach is the significant expense associated with implementing Condition-Based or Predictive maintenance strategies, coupled with a lack of knowledge about the modelling or behaviour of a failing system.

In most facilities, maintenance continues to be performed according to a predefined schedule. An optimization of this approach involves intervening in the system only when necessary, which requires the knowledge of when a system is malfunctioning. Fault Detection and Novelty Detection enable triggering an event when a known fault occurs or when a new, unfamiliar behaviour emerges in the maintained system. 

In this thesis, a \gls{glo:frmwrk} that performs Novelty Detection is proposed. The structure of the \gls{glo:frmwrk} is thought to be modular and general-purpose to ease the implementation into different systems. It is developed following an Unsupervised Machine Learning approach to overcome the common lack of physical models of the maintained device. The Machine Learning core of the \gls{glo:frmwrk} is based on the \gls{glo:feature}s extracted from the data gathered from sensors. In the first phase, the data are used to train the models. Then, the \gls{glo:frmwrk} operates in real time, continuously assessing the status of the system. This solution provides a novelty metric that estimates how unfamiliar the current state of the system is and a forecast of the future evolution of the system.

Firstly, it has been developed to be executed and tested on a \gls{pc} using various Unsupervised Machine Learning algorithms. The algorithm that appeared to better balance performance and hardware resource consumption was deployed on a microcontroller. The proposed solution includes all the tools necessary in the data pipeline. Relying on the general-purpose structure proposed, the \gls{glo:frmwrk} can be easily set up on a machine and extended to an arbitrary configuration of sensors and \gls{glo:feature}s. 

The \gls{pc} implementation underwent testing using various Unsupervised algorithms on publicly available datasets, while the edge implementation was tested through laboratory experiments.

Both the tests on datasets and the experimental results showed that the proposed \gls{glo:frmwrk} is able to detect novelties and give an estimate of the future evolution of the novelty metric of the system.