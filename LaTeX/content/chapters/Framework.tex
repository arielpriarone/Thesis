\chapter{Proposed Framework}
\label{ch:Framework}

In the \autoref{ch:FeatureExtraction} the features extraction process has been described. Before approaching the problem of performing \gls{nd} in \gls{glo:edge}, let's build a framework in \texttt{python} that runs on a \gls{pc} that is \emph{configurable}, \emph{modular}, \emph{expandable} and \emph{general purpose}. This framework will be used to test the features extraction process and to test the \gls{ml} algorithms before selecting one of them to be implemented \gls{glo:edge} framework, which will be harder to configure.

A real case application would probably have several signals of several physical quantities, so a general approach that can manage different types of features, and extract from each of them the most relevant information, is needed.

The proposed framework is thought to be set up on any type and combination of sensors. The framework is thought to manage data that are correlated to a specific fault. For example, think about a \gls{cnc} machine like the one in \autoref{fig:cnc}. It has five axes, so a solution would be to instance the framework five times, one for each axis, linked to vibration sensors, temperature sensors etc. of the considered axis. This would allow us to pinpoint the fault to a specific axis. Another concern is, what if the single axis is seeing a normal condition, but the machine as a whole is not? This may happen if the tool has a problem: the vibration registered in the spindle would be normal in general, but would not be normal \emph{related} to the feed rate that another axis is imposing. To address this scenario, other instances of the framework can be set up, that also receive the speeds from the machine controller and the feed rate from the \gls{cnc} program as well as data from the sensors used in the other instances. This would allow us to detect a more complex fault, that is not characteristic of any part of the machine, but of the machine as a whole. The former kinds of instances would allow specific faults to be detected, giving also an idea of what the fault is, the latter would allow complex faults to be detected, but would not give a precise idea of what the fault is. The framework is thought to be able to manage both of these scenarios, and to be able to manage them together.

\begin{figure}
  \centering
  \includegraphics[width=.7\textwidth]{images/Framework/millmachine.jpg}
  \caption{A 5-axis \gls{cnc} milling machine. \cite{FagorAutomation}}
  \label{fig:cnc}
\end{figure}

\section{Commissioning}
\label{sec:commissioning}
To adapt the framework to a specific machine, the \gls{glo:commissioning} of the \gls{ml} system would have to be done in steps. Starting from the data acquisition and ending with the predictions of \gls{rul} and model updates, the steps are described in this section.

\subsection{Data structure}
\label{sec:Data_structure}
The first phase of adaptation of the framewor to a machine is to define what data to sample and how to sample them. This includes the decision of which sensors to use, the sampling frequencies, the data acquisition system and which features are needed to be extracted from each sensor data. At this point, if more than one instance of the framework is needed, the sets of sensors and features to be used in each instance are defined. For example, in a shaft with two bearings, each with two accelerometers, the first instance of the framework would be linked to the first bearing, and would use the data from the two accelerometers to extract the features that are needed to detect the fault in the first bearing. The second instance of the framework would be linked to the second bearing, and would use the data from the other two accelerometers to extract the features that are needed to detect the fault in the second bearing. Optionally, a third instance of the framework would use the data from all four accelerometers to detect a generic fault in the shaft.
Those decisions influence the structure of the database, which is described in \autoref{sec:Database}. 

\subsection{Data acquisition}
Once the structure of the data is defined, the first phase of the \gls{glo:commissioning} procedure is to set up the data acquisition. This has to be done when the machine is new or, at least, someone guarantees that the machine is in an healthy condition.

During the previous phase, the number of instances of the framework are defined. Each instance would have it's own database. This phase is just a metter of storing the data that will be used to train the models the first time. A software agent, that we call \gls{glo:fieldagent}, is responsable for this task. 
This phase last until the database is filled with enough data to train the models.

\subsection{Training}
The second phase of the \gls{glo:commissioning} procedure is to train the models. Once the healthy data are enough to characterize all the normal conditions of the maintaned system, all the recorded data are elaborated by another software agent that we call \gls{glo:featAg} (\gls{fa}). This agent extract all the features from the time-series and store them in a structured way.

Once all the features are available in the database, another agent called \gls{glo:mla} (\gls{mla}) is responsable to train the models. All the models considered are \gls{uml} models. The models are trained on a standardized version of the feature matrix. The standardization is done \gls{wrt} the time evolution, \gls{ie} all the features used for training have a time evolution with zero mean and unit variance. This is done because most \gls{ml} algorithms are sensitive to the scale of the features.

All that has been said is valid for a single instance of the framework. If more than one instance is needed, the training phase has to be done for each instance.

\subsection{Evaluation}
At this point, a model that represents the normal condition of the system is available (actually a model for each instance of the framework). The next step is to evaluate the model.

In this phase, the machine continue to perform its normal operations. The \gls{glo:fieldagent} provides the sampled data, the \gls{glo:featAg} extracts the features and the \gls{glo:mla} evaluate the health of the system.
The proposed novelty/fault metric and procedure are specific to the model used, as described in the dedicated chapter about \gls{uml} models (\autoref{sec:clust_metric} for the k-means, \autoref{dbscan_eval} for \gls{dbscan}, \autoref{sec:gauss_eval} for \gls{gmm}, \autoref{sec:svm_eval} for \gls{nu_svm}, \autoref{sec:iforest_eval} and \autoref{sec:lof_eval} for \gls{lof}). 

Now the \gls{nd} is up and running. The novelty metric is used to decide if the system is healthy or not. The metric is plotted to the user to see.
Note that in a classic \gls{ml} approach, the dataset is split into a training set and a test set. In this case the test set is the data that are sampled during the evaluation phase. It's equivalent to say that the model is trained on the past data and evaluated on the future data, or that the framework works in testing phase for an undetermined amount of time, until the user decide to update the models. This is equivalent to a test phase because if during this phase the framework outputh too many false positives, the user will decide to update the models. Otherwise, it means that the models are working properly and this phase can last indefinitely.

\subsection{Model update}
Once the metric overshoots the threshold, the \gls{mla} warns the user that the system is not healthy, and start to perform \gls{pdm} predicting the future evolution of the metric and the \gls{rul} of the system. 
Again, this condition can last indefinitely, once new data are sampled, the \gls{mla} evaluates the health of the system and updated the predictions.

It's up to the user to determine if the warning is a false positive or a real fault. In the former case, the user can decide to command the \gls{mla} to update the models. The snapshots that generated the the warning are incorporated in the training set, and the models are retrained, returning to the evaluation phase. In the latter case, the user can simply perform the repairs/maintanence needed and restore the system to a healthy condition or can use the snapshots that generated the warning to train a new model that represents the fault condition.

If the user decides to train also the second model with the snapshot declared faulty, the system returns into the evaluation phase, but now the \gls{mla} outputs two metrics: one estimate the health of the system and the other how similar the behavior of the system is compared to any known fault condition.






\section{Database}
\label{sec:Database}
In the previous \autoref{sec:commissioning}, the setup behaviour phases of the framework have been described, referring to a generic \quoted{database}, without specifying the structure of the database. Let's now address the problem of storing the data efficiently and effectively. Instead of relying on \texttt{python} data structures, it is better to use a dedicated database manager.

The proposed framework uses \gls{glo:mongodb} for the following reasons. It is a widely-used, open-source \gls{glo:nosql} database that is designed to handle unstructured or semi-structured data. It utilizes a document-oriented data model, storing data in flexible, \gls{json}-like \gls{bson} format. MongoDB is suitable for implementation in a \gls{nd} framework due to its scalability, flexibility, and real-time data processing capabilities. In novelty detection, the system often deals with diverse and dynamic data sources, making MongoDB's \quoted{unstructureness} advantageous for handling varying data formats and evolving data requirements. It has the ability to handle large volumes of data and support scaling allowing for efficient storage and retrieval of information in real-time, crucial for real-time applications. Moreover, MongoDB has a rich query language and secondary indexes that allow for fast and efficient querying of data and a library for \texttt{python} that makes it easy to use.
The \gls{json} format is also human-readable, which makes it easy to understand the data stored in the database, and \quoted{mongoDB Compass} is a graphical user interface that allows one to easily explore the database.

\subsection{Collections}
\begin{longtblr}[
  caption = {Collections contained in the \gls{glo:mongodb} database},
  label = {tab:MongoDB_collections},
  ]{
  hline{1,11} = {-}{0.08em},
      hline{2} = {-}{},
    }
  \textbf{Collection} & \textbf{Content}                                       \\
  raw                 & time-series and information about them                 \\
  unconsumed          & snapshots to be evaluated                              \\
  quarantined         & {snapshots detected as novelty waiting to be declared  \\healthy, faulty or be discarded}\\
  healthy             & snapshots declared as normal behaviour                 \\
  healthy train       & {training dataset (scaled, processed, packet)          \\for the \gls{nd} \gls{uml} model}\\
  faulty              & snapshots declared as faulty behaviour                 \\
  faulty train        & {training dataset (scaled, processed, packet)          \\for the \gls{nd} \gls{uml} model}\\
  models              & {models trained on healthy and faulty data the metrics \\and predictions to be shown}\\
  backup              & time-series, features, models, etc.
\end{longtblr}

MongoDB structure is based on collections, which are groups of (\gls{json}) documents. A document is a set of key-value pairs that can be nested in several layers. Documents have a dynamic schema, which means that documents in the same collection do not need to have the same set of fields or structure, and common fields in a collection's documents may hold different types of data. To store the data needed by the framework the collections reported in \autoref{tab:MongoDB_collections} are used.
In the following paragraphs, the structure and purposes of each collection are described.

\paragraph{Raw}
Thinking about the data flow, the first interface between the hardware and the software would be the sensor readings. Every sensor should have a name and be sampled at a constant frequency (or, at least, the sensors that provide data for frequency-domain feature extraction should have a constant sampling frequency). This data is stored in the {raw} collection, with the \gls{json} structure summarized in \autoref{tab:raw_json}.
\begin{longtable}{lll}
  \caption{Structure of the \quoted{raw} collection \gls{json} configuration file.}\\ 
  \toprule
  \textbf{Field} & \textbf{Sub-Field} & \textbf{Type} \endfirsthead 
  \hline
  \texttt{\textunderscore}id &  & string \\
  timestamp &  & ISO date \\*
  \multirow{2}{*}{sensor 1} & sampling frequency & float \\*
   & time-serie & list[float] \\*
  \multirow{2}{*}{sensor 2} & sampling frequency & float \\*
   & time-serie & list[float] \\
  $\dots$ & $\dots$ & $\dots$ \\*
  \multirow{2}{*}{sensor~$n$} & sampling frequency & float \\*
   & time-serie & list[float] \\
  \bottomrule
  \end{longtable}
where \texttt{\textunderscore id} is the unique identifier of the document, \texttt{timestamp} is the time at which the data was acquired, in \gls{iso} format, and \texttt{Sensor 1} to \texttt{Sensor n} are the names of the sensors. Each sensor has a \texttt{sampFreq} field that contains the sampling frequency of that particular sensor, and a \texttt{timeSerie} field that contains the data acquired by the sensor, as a list. The \texttt{timeSerie} field is a list of floating point numbers, that can be of any length. Note that the sampling frequencies of different sensors can be different, for example, if a timestamp contains $1\si{\s}$ period of data, a vibration sensor would be linked to an array with several thousands of samples, while a temperature sensor would be linked to only one sample.

\paragraph{Unconsumed}
Once defined the structure that the time-series will have in the database, let's define the structure of the snapshots. The features extracted from the time-series are stored in the {unconsumed} collection, with the \gls{json} structure described in \autoref{tab:unconsumed_json}.

\begin{longtable}{lll}
  \caption{Structure of the \quoted{unconsumed} collection \gls{json} configuration file.}\label{tab:unconsumed_json}\\ 
  \toprule
  \textbf{Field} & \textbf{Sub-Field} & \textbf{Type} \endfirsthead 
  \hline
  \texttt{\textunderscore}id & - & string \\
  timestamp & - & ISO date \\
  \vcell{sensor 1} & \vcell{mean} & \vcell{float} \\*[-\rowheight]
  \printcelltop & \printcellmiddle & \printcellmiddle \\
   & root mean square & float \\
   & peak to peak & float \\
   & standard deviation & float \\
   & skewness & float \\
   & kurtosis & float \\
   & wavelet coefficient~$1$ & float \\
   & wavelet coefficient~$2$ ~ & float \\
   & $\vdots$ & $\vdots$ \\
   & wavelet coefficient~$2^{\text{three dept}}$ & float \\
  sensor 2 & mean & float \\
   & root mean square & float \\
   & peak to peak & float \\
   & standard deviation & float \\
   & skewness & float \\
   & kurtosis & float \\
   & wavelet coefficient~$1$ & float \\
   & wavelet coefficient~$2$ & float \\
   & $\vdots$ & $\vdots$ \\
   & wavelet coefficient~ $2^{\text{three dept}}$~~ & float \\
  $\vdots$ & $\vdots$ & $\vdots$ \\
  sensor~$n$ & mean & float \\
   & root mean square & float \\
   & $\vdots$ & $\vdots$ \\
   & wavelet coefficient~ ~~$2^{\text{three dept}}$ & float \\
  novelty evaluated flag & - & boolean \\
  \bottomrule
  \end{longtable}
  
Notice that different sensors can have different features. The \quoted{novelty evaluated} field is a boolean that is set to \texttt{false} when the snapshot is created, and is set to \texttt{true} when the \gls{nd} algorithm evaluates the snapshot. This field is used to avoid evaluating the same snapshot multiple times while leaving it in the collection until also the \gls{fd} algorithm is performed. At this point, the snapshot will be moved either to the backup collection, discarded or to the quarantine collection if either the \gls{nd} or the \gls{fd} flag it.

\paragraph{Quarantined}
The \quoted{quarantined} collection is used to store the snapshots that were flagged as \quoted{novelty} by the \gls{nd} algorithm or as \quoted{faulty} by the \gls{fd} algorithm (or were flagged by both of them). The structure is the same as the \quoted{unconsumed} collection, but the \quoted{novelty evaluated} field is not present since, at this point, the snapshots are guaranteed to have been evaluated. The snapshots in this collection are waiting to be declared as \quoted{healthy} or \quoted{faulty} by the user or to be discarded.

\paragraph{Healthy}
The idea behind the \quoted{healthy} collection is to store the snapshots that are acquired during the first work phase of the framework, before training, or the snapshots that were in the \quoted{quarantine} collection and were declared as healthy by the user. The documents in this collection have the same structure as the documents in the \quoted{quarantined} collection.

\paragraph{Healthy train}
In this collection the healthy snapshots are packed together in different documents, each of them useful in a different phase of the training process.

{The first document has the \texttt{id} \texttt{training\textunderscore set}, that contains all the \texttt{N} training snapshots, each of them with \texttt{n} sensors signals, characterized by \texttt{F} features. For ease of accessibility, every bottom-nested field is a list of \texttt{N} elements. The structure is resumed in \autoref{tab:train_json}.}


\begin{longtable}{lll}
\caption{Structure of the \quoted{healthy train} collection \gls{json} configuration file.}\label{tab:train_json}\\ 
\toprule
\textbf{Field} & \textbf{Sub-Field} & \textbf{Type} \endfirsthead 
\hline
\texttt{\textunderscore}id & - & string \\
timestamp & - & list[ISO date] \\
\vcell{sensor 1} & \vcell{feature 1} & \vcell{list[float]} \\*[-\rowheight]
\printcelltop & \printcellmiddle & \printcellmiddle \\
 & feature 2 & list[float] \\
 & $\vdots$ & $\vdots$ \\
 & feature F & list[float] \\
sensor 2 & feature 1 & list[float] \\
 & feature 2 & list[float] \\
 & $\vdots$ & $\vdots$ \\
 & feature F & llist[loat] \\
$\vdots$ & $\vdots$ & $\vdots$ \\
sensor~$n$ & feature 1 & list[float] \\
 & feature 2 & llist[loat] \\
 & $\vdots$ & $\vdots$ \\
 & \begin{tabular}[c]{@{}l@{}}feature F\\\end{tabular} & list[float] \\
\bottomrule
\end{longtable}


This collection contains other three documents:
\begin{itemize}
  \item \texttt{training set scaled}, that contains the scaled training set, having the same structure as the \texttt{training set} document;
  \item \texttt{training set MIN MAX}, that contains the minimum and maximum values of the features of the training set, useful to plot the features with a reference of the bounds of the training set. It has the same structure of the \texttt{training set} document, but the bottom-nested fields are lists of two elements (the minimum and the maximum value);
  \item \texttt{StandardScaler\textunderscore pickled}. It contains the \texttt{StandardScaler} object that was used to scale the training set. This object is encoded in \gls{glo:pickle}, and it is used during the evaluation phase to scale the snapshots before evaluating them.
\end{itemize}

\paragraph{Faulty}
This collection serves the same exact purpose as the \quoted{healthy} collection, but for the faulty snapshots. Faulty snapshots are not discarded because they can be used to train the \gls{fd} \gls{uml} algorithm.

\paragraph{Faulty train}
This collection serves the same exact purpose as the \quoted{healthy train} collection, but for the faulty snapshots.

\paragraph{Models}
This collection contains the models trained on the healthy and faulty data and a buffer of the predictions and metrics to be displayed to the user.

The structure of the models' documents is just an identifier and the \texttt{python} object of the model, encoded in \gls{glo:pickle}. The structure of the predictions and metrics documents is the \autoref{tab:model_json}:

\begin{longtable}{lll}
  \caption{Structure of the \quoted{models} collection \gls{json} configuration file.}\label{tab:model_json}\\ 
  \toprule
  \textbf{Field} & \textbf{Sub-Field} & \textbf{Type} \endfirsthead 
  \hline
  \texttt{\textunderscore}id & - & string \\
  timestamp & - & list[ISO date] \\
  values & - & list[float] \\
  assigned cluster & - & list[int] \\
  anomaly flag & - & list[bool] \\
  prediction curve parameters & - & pickle format \\
  \bottomrule
  \end{longtable}

\paragraph{Backup}
The backup collection is a general-purpose container for any document that needs to be stored for backup purposes. It can contain time-series, features, models, etc. The structure of the documents in this collection is the same as the structure of the documents in the other collections.

