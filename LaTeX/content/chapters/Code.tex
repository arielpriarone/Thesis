\chapter{Code}
\label{ch:code}
Including all the code used for the development of this thesis is not feasible. However, the complete code is available on GitHub at the link shown in the introduction. In this appendix, the code of just some of the agents methods is shown. The code is written in \gls{glo:python}.
\section{Feature Agent (\gls{fa})}
\label{sec:fa}
\begin{minted}[linenos,tabsize=2,breaklines]{python}
class FA(src.data.DB_Manager): # feature agent
    '''
    empty the RAW collection and populate the Unconsumed collection with extracted features:
    '''
    def __init__(self, configStr: str, order: int = 1):
        super().__init__(configStr)
        if order not in [-1,1]:
            raise ValueError('order must be either latest or oldest')
        self.order  =   order                       # pick -1=latest / 1=oldest raw data available
        self.__last_snap_timestamp = None           # timestamp of the last snapshot plotted
    
    def _readFromRaw(self):
        ''' Read the data from the RAW collection '''
        try:
            self.snap = self.col_raw.find(). sort('timestamp',self.order).limit(1)[0]     # oldest/newest record - sort gives a cursor, the [0] gets the dicttionary
            print(f"Imported snapshot with timestamp {self.snap['timestamp']} from {self.col_raw}")
            return True    
        except IndexError:
            print(f"No data in collection {self.col_raw.full_name}, waiting for new data...")
            return False
    def _extractFeatures(self):
        ''' extract features from the data '''
        for sensor in self.sensors:                                  # for each sensor (names are keys of the dict)
            self.features["timestamp"] = self.snap["timestamp"]      # add the timestamp to the features
            self._extractTimeFeautures(sensor)                       # extract time domain features
            self._extractFreqFeautures(sensor)                       # extract frequency domain features
    def _extractTimeFeautures(self, sensor):
        ''' extract time domain features '''
        # if Mean Enabled
        if self.Config['Database']['sensors'] [sensor]['features']['mean']:
            self.features[sensor].update({'mean': np.mean(self.snap[sensor]['timeSerie'])})
            print(f"Mean extracted from [purple]{sensor}[/]")
        # if RMS Enabled
        if self.Config['Database']['sensors'] [sensor]['features']['rms']:
            self.features[sensor].update({'rms': np.sqrt(np.mean(np.square(self.snap[sensor] ['timeSerie'])))})
            print(f"RMS extracted from [purple]{sensor}[/]")
        # if peak2peak Enabled
        if self.Config['Database']['sensors'] [sensor]['features']['peak']:
            self.features[sensor].update({'peak2peak': np.ptp(self.snap[sensor]['timeSerie'])})
            print(f"Peak2Peak extracted from [purple]{sensor}[/]")
        # if std Enabled
        if self.Config['Database']['sensors'] [sensor]['features']['std']:
            self.features[sensor].update({'std': np.std(self.snap[sensor]['timeSerie'])})
            print(f"Standard deviation extracted from [purple]{sensor}[/]")
        # if skewness Enabled
        if self.Config['Database']['sensors'] [sensor]['features']['skew']:
            self.features[sensor].update({'skewness': scipy.stats.skew(self.snap[sensor]['timeSerie'])})
            print(f"Skewness extracted from [purple]{sensor}[/]")
        # if kurtosis Enabled
        if self.Config['Database']['sensors'] [sensor]['features']['kurt']:
            self.features[sensor].update({'kurtosis': scipy.stats.kurtosis(self.snap[sensor]['timeSerie'])})
            print(f"Kurtosis extracted from [purple]{sensor}[/]")
    def _extractFreqFeautures(self, sensor):
        # if Wavelet is enabled
        if self.Config['Database']['sensors'] [sensor]['features']['wavPowers']:    
            _, pows, nodes, _, _ = packTrasform(self.snap[sensor]['timeSerie'],     # perform the wavelet trasform
                wavelet=self.Config["wavelet"]["type"],
                mode=self.Config["wavelet"]["mode"],
                maxlevel=self.Config["wavelet"]["maxlevel"], 
                plot=False)
            self.features[sensor].update(dict(zip(nodes, pows)))  # create a dictionary with nodes as keys and powers as values
            print(f"Wavelet coefs extracted from [purple]{sensor}[/]")
                
    def _deleteFromraw(self):
        ''' delete current snap record from the RAW collection '''
        self.col_raw.delete_one({'_id':self.snap['_id']})
        print(f"Deleted snapshot with timestamp {self.snap['timestamp']} from {self.col_raw}")
    def _writeToUnconsumed(self):
        ''' write the extracted features to the Unconsumed collection '''
        __dummy=self.features.copy() # create a copy of the features dictionary
        self.col_unconsumed.insert_one(__dummy) # insert the features in the Unconsumed collection, without changing the dictionary
    def initialize_barPlotFeatures(self,axs: plt.Axes):
        """
        Initializes the bar chart of the latest features for each sensor in the collection.
        """
        src.vis.set_matplotlib_params()
        try:
            snap = self.col_unconsumed.find().sort('timestamp', pymongo.DESCENDING).limit(1)[0]  # latest document in collection
        except IndexError:
            print('No data in collection, wait for new data...')
            return None
        try:
            self.MinMax = self.col_healthy_train.find({'_id': 'training set MIN/MAX'})[0]
        except IndexError:
            self.MinMax = None
        tab10_cmap = cm.get_cmap("Set1")
        self.__colors = [tab10_cmap(indx) for indx, _ in enumerate(self.sensors)] # convert tuple to list
        self.__base_width = 1.0     # the width of the bars
        self.__separator  = 0.5   # the space between the bars
        self.__features_list = []  # list of all features
        self.__Scaler = src.models.MLA.retrieve_StdScaler( col=self.col_healthy_train)
        for sensor in self.sensors:
            self.__features_list.append(list(snap[sensor].keys()))
        self.features_list = list(chain.from_iterable(self.__features_list))  # flatten list
        self.features_list = list(dict.fromkeys(self.features_list))  # remove duplicates
        self.__feature_mask = {key: [False] * len(self.sensors) for key in self.features_list} # initialize dictionary
        for sensor_number, sensor in enumerate(self.sensors):
            for feature in self.features_list:
                if feature in snap[sensor].keys():
                    self.__feature_mask[feature][sensor_number] = True
        self.__locator_bars = [0.0]  # the x locations for the groups
        self.__locator_ticks = []  # the x locations for the ticks
        self.__minMax = [0.0,0.0]
        for feature in self.features_list:        
            for sensor_number, sensor in enumerate(self.sensors):
                if self.__feature_mask[feature][sensor_number] and  self.MinMax is not None:                        
                    self.__minMax[0] = min(self.__minMax[0], self.MinMax[sensor][feature][0])
                    self.__minMax[1] = max(self.__minMax[1], self.MinMax[sensor][feature][1])
        
        for feature in self.features_list:
            width = self.__base_width
            offset = 0.0          
            for sensor_number, sensor in enumerate(self.sensors):
                if self.__feature_mask[feature][sensor_number]:                        
                    offset += width
            self.__locator_ticks.append(self.__locator_bars[-1] + (offset-width) / 2 if offset > 0 else self.__locator_bars[-1])
            self.__locator_bars.append(self.__locator_bars[-1] + offset + self.__separator)
        self.__legend_lines = [Line2D([0], [0], color=self.__colors[indx], lw=4, label=sensor) for indx, sensor in enumerate(self.sensors)] # type: ignore
        self.__legend_labels = copy.deepcopy(self.sensors)
        if self.MinMax is not None:
            self.__legend_lines.extend([Line2D([0], [0],marker=6, color='w',markerfacecolor=self.__colors[indx], lw=4, alpha=1) for indx, sensor in enumerate(self.sensors)]) # type: ignore
            minmax_legend = [f"{sensor} min/max" for sensor in self.sensors]
            self.__legend_labels.extend(minmax_legend)
            axs.set_ylim(ymin=self.__minMax[0]*1.1, ymax=self.__minMax[1]*1.3) # type: ignore
        return axs
    def barPlotFeatures(self,axs: plt.Axes):
        """
        Plots a bar chart of the latest features for each sensor in the collection.
        Parameters:
        axs (matplotlib.axes.Axes): The axes on which to plot the bar chart.
        Returns:
        matplotlib.axes.Axes: The axes on which the bar chart was plotted.
        """
        try:
            snap = self.col_unconsumed.find().sort('timestamp', pymongo.DESCENDING).limit(1)[0]  # latest document in collection
        except IndexError:
            print('No data in collection, wait for new data...')
            return
        if snap['timestamp'] == self.__last_snap_timestamp:
            print('Latest data already plotted... waiting for new data...')
            return
        axs.clear()  # Clear last data frame
        axs.set_title(f"Latest features for each sensor. Timestamp: {snap['timestamp']}")  # set title
        if self.__Scaler is not None:      # if the scaler is available, scale the data
            for sensor in self.sensors:
                _data_to_scale = np.array(list(snap[sensor].values())). transpose().reshape(1,-1)
                _data_scaled = self.__Scaler[sensor].transform(_data_to_scale). transpose().tolist()
                snap[sensor] = dict(zip(list(snap[sensor].keys()), _data_scaled))
                axs.set_title(f"Latest standardized features for each sensor. Timestamp: {snap['timestamp']}")
    
        for bar_group, feature in enumerate(self.features_list):
            width = self.__base_width
            offset = 0.0          
            for sensor_number, sensor in enumerate(self.sensors):
                if self.__feature_mask[feature][sensor_number]:                        
                    axs.bar(self.__locator_bars[bar_group]+offset, snap[sensor][feature], width, color=self.__colors[sensor_number], alpha=1)
                    if self.MinMax is not None:
                        axs.scatter(self.__locator_bars[bar_group] +offset,self.MinMax[sensor][feature][0], marker=6, s=15, color=self.__colors[sensor_number], alpha=1) # type: ignore
                        axs.scatter(self.__locator_bars[bar_group] +offset,self.MinMax[sensor][feature][1], marker=7, s=15, color=self.__colors[sensor_number], alpha=1) # type: ignore
                    offset += width
        axs.set_xticks(self.__locator_ticks,self.features_list)
        axs.tick_params(axis='x',rotation = 90)
        axs.legend(self.__legend_lines, self.__legend_labels, loc='upper right',  ncol=len(self.sensors))
        axs.set_ylabel('Feature value [-]')
        axs.set_xlabel('Features [-]')
        axs.spines['left'].set_visible(True)
        axs.spines['bottom'].set_visible(True)
        axs.grid(True,which='both',axis='x')
        self.__last_snap_timestamp = snap['timestamp']
        plt.tight_layout()
        if __name__=='__main__':
            plt.show()
        return axs
                
    def run(self):
        while True:
            os.system('cls')
            while not self._readFromRaw(): pass  # wait for new data
            self._extractFeatures()
            self._writeToUnconsumed()
            self._deleteFromraw()    
\end{minted}
\section{Machine Learning Agent (\gls{mla})}
\label{sec:mla}
\begin{minted}[linenos,tabsize=2,breaklines]{python}
class MLA(src.data.DB_Manager):
    '''
    Machine Learning Agent:
    '''
    def __init__(self, configStr: str, type: str = 'novelty'):
        super().__init__(configStr)
        self.type = type              #  type of the MLA (novelty/fault) - how normal/how faulty the data are
        self.__max_clusters = self.Config['kmeans']['max_clusters']
        self.__max_iter = self.Config['kmeans']['max_iterations']
        self.__error_queue_size = self.Config['kmeans']['error_queue_size']
        self.__error_plot_size = self.Config['kmeans']['error_plot_size']
        if self.__error_queue_size > self.__error_plot_size:
            raise ValueError('Error queue size cannot be bigger than the error plot size in "config.yaml"')
        self.novelty_threshold = self.Config['novelty']['threshold']
        self.forecast_threshold = self.Config['novelty']['forecast_threshold']
        self.outlier_filter = self.Config['novelty']['outlier_filter']
        self.n_fit = self.Config['novelty']['n_fit']
        if self.n_fit > self.__error_plot_size:
            raise ValueError('N_fit cannot be bigger than the error plot size in "config.yaml"')
        self.regType = self.Config['novelty']['regressor']
        self.err_dict = {'values': List[float], 'timestamp': List[datetime.datetime],
                         'assigned_cluster': List[int], 'anomaly': List[bool]} # dictionary of the error
        self.err_dict['values'] = []  # initialize the error array
        self.err_dict['timestamp'] = [] # initialize the timestamp array
        self.err_dict['assigned_cluster'] = []  # initialize the assigned cluster array
        self.err_dict['anomaly'] = []# initialize the anomaly array
        self.err_dict['pred_parameters'] = [] # initialize the pred_parameters array
        self.__mode = None              #  mode of the MLA (evaluate/train/retrain)
        match self.type:
            case 'novelty':
                self.col_features = self.col_healthy
                self.col_train = self.col_healthy_train
            case 'fault':
                self.col_features = self.col_faulty
                self.col_train = self.col_faulty_train
            case _:
                raise ValueError('Type of MLA is not valid. It should be either "novelty" or "fault", but it is: ' + self.type)
        try:
            self.__retrieve_StdScaler() # retrieve the scaler
        except:
            self.StdScaler: Dict[str, StandardScaler] = {} # if the scaler is not found, initialize it
        try:
            self.retrieve_KMeans() # retrieve the Kmeans model
        except:
            self.kmeans=KMeans()
        # logger init
        logging.basicConfig(filename=os.path.join(self.Config ['miscellanea']["logpath"],'MLA.log'), filemode='w', format='%(asctime)s - %(message)s', level=logging.INFO)
    @property
    def mode(self):
        return self.__mode
    
    @mode.setter
    def mode(self, value: str):
        if value not in ['evaluate', 'train', 'retrain']:
            print('Mode not valid. It should be either "evaluate", "train" or "retrain", but it is: ' + value)
        else:
            self.__mode = value
    def run(self):
        '''Run the MLA according to its state'''
        while True:
            os.system('cls')
            match self.mode:
                case 'evaluate':
                    self.evaluate()
                case 'train':
                    if typer.confirm(f"The training procedure will take all the data from the collection '{self.col_features.full_name}' and pack it in the collection '{self.col_train.full_name}'. This will also ERASE the current training data, do you want to PROCEED?", abort=True):
                        self.col_train.delete_many({})
                    if self.col_features.count_documents({}) == 0:
                        self.__move_to_train(source=self.col_ unconsumed) # empty, ask to move all data from unconsumed to train dataset
                    while not self.prepare_train_data():
                        pass    # wait for data to be available
                    self.train()
                    if typer.confirm("Do you want to change the 'mode' to 'evaluate'", abort=True):
                        self.mode = 'evaluate'
                case 'retrain':
                    if self.col_features.count_documents({}) == 0:
                        self.__move_to_train(source=self.col_ quarantined) # empty, ask to move all data from unconsumed to train dataset
                    while not self.prepare_train_data():
                        pass    # wait for data to be available
                    self.train()
                    if typer.confirm("Do you want to change the 'mode' to 'evaluate'", abort=True):
                        self.mode = 'evaluate'
                case _:
                    self.mode = typer.prompt('Please select the mode of the MLA. The options are: "evaluate", "train" or "retrain"')
    def evaluate(self):
        self.retrieve_KMeans()      # retrieve the Kmeans model
        self.num_clusters = self.kmeans.get_params()['n_clusters'] # get the number of clusters
        self.packFeaturesMatrix()      # pack the training features in a matrix
        while True:
            self.calculate_train_cluster_dist() # calculate the maximum distance of each cluster in the train dataset
            evaluate=False
            printed=False
            while not evaluate: # read the features from the collection
                try:
                    self.snap=self.col_unconsumed.find({f'{self.type} evaluated': {"$exists": False}}). sort('timestamp', pymongo.ASCENDING). limit(1)[0] # get the oldest not evaluated snap
                    evaluate=True
                except IndexError:
                    if not printed:
                        print(f"No data to evaluate in the '{self.col_unconsumed.full_name}' collection, waiting for new data...")
                        printed=True
            self.scale_features()
            if self.evaluate_error():      # evaluate the error - if novelty detected, move to quarantine
                if self.type == 'novelty':
                    self._find_snap(self.snap["_id"], self.col_unconsumed) # find the snap in the features collection (to preserve unscaled version)
                    self._write_snap(self.col_quarantined) # move the snap to the quarantine collection
                self.predict() # predict the fault
            self._mark_snap_evaluated() # mark the snap as evaluated
            if self.type == 'novelty':
                self._delete_evaluated_snap() # delete the snap from the unconsumed collection
            print(f"Distance Novelty: {self.err_dict['values'][-1]}")
    
    def predict(self):
        print("Predicting the fault...")
        ''' This method predicts the fault '''
        if len(self.err_dict['values']) < src.data.f.__code__.co_argcount:
            print("Not enough data to predict the fault")
            return
        start_fit = min(len(self.err_dict['timestamp']),self.n_fit) # start of the error samples to fit
        range_to_fit = range(-start_fit,0) # range of the error samples to fit
        x = np.array([self.err_dict['timestamp'][i].timestamp() for i in range_to_fit])
        xscale = float(max(x)-min(x))
        xoffset = float(min(x))
        x = (x-xoffset)/xscale # scale the x axis
        y = np.array(self.err_dict['values'][-self.n_fit:])
        params = None
        match self.regType:
            case 'exp':
                params = src.ExpRegressor(x,y) #fitting
            case 'scipy':
                try:
                    params, cv = opt.curve_fit(src.data.f, x, y) #fitting
                except:
                    pass
        if params is None:
            print("Error in the fitting procedure of the prediction curve")
            return
        __pickled_data = pickle.dumps([xoffset, xscale, params])
        print(f"Fault predicted with parameters: {params}")
        self.err_dict['pred_parameters'].append(__pickled_data)
        self.err_dict['pred_parameters'] = self.err_dict['pred_parameters'][-self.__error_queue_size:]
    def _mark_snap_evaluated(self): # to leave at least one snap in the collection for plotting reasons
        self._find_snap(self.snap["_id"],self.col_unconsumed) # find the snap in the features collection (to preserve unscaled version)
        self.snap[f'{self.type} evaluated'] = True
        self._replace_snap(self.col_unconsumed) # mark the snap as evaluated
        print(f"Snap '{self.snap['_id']}' marked as evaluated in the '{self.col_unconsumed.full_name}' collection")
    
    def _delete_evaluated_snap(self): # to leave at least one snap in the collection for plotting reasons
        while self.col_unconsumed.count_documents({'novelty evaluated': True, 'fault evaluated': True}) > 1: # while there are more than one snap to delete
            snap_to_delete = self.col_unconsumed.find({'novelty evaluated': True, 'fault evaluated': True}).sort('timestamp',pymongo.ASCENDING).limit(1)[0] # get the oldest snap to delete
            self.col_unconsumed.delete_one({'_id': snap_to_delete['_id']}) # delete the snap from the collection
            print(f"Snap '{snap_to_delete['_id']}' deleted from the '{self.col_unconsumed.full_name}' collection")
    def scale_features(self):
        for sensor in self.sensors:
            _data_to_scale = np.array(list(self.snap[sensor].values())).transpose(). reshape(1,-1)
            _data_scaled = self.StdScaler[sensor].transform(_data_to_scale). tolist()[0]
            self.snap[sensor] = dict(zip(list(self.snap[sensor].keys()), _data_scaled))
    def evaluate_error(self):
        _features_values = []
        for sensor in self.sensors:
            _features_values.append([float(value) for value in self.snap[sensor].values()])
        _features_values_flat = [item for sublist in _features_values for item in sublist] # flatten the list
        y=self.kmeans.predict(np.array(_features_values_flat). reshape(1,-1)) # predict the cluster for the new snap
        distance_to_assigned_center = self.kmeans.transform(np.array(_features_values_flat). reshape(1,-1))[0,y]
        # the actual estimator of the error is the relative distance margin to the assigned cluster
        current_error=float((distance_to_assigned_center - self.train_cluster_dist[int(y)]) / self. train_cluster_dist [int(y)]) # calculate the error
        if self.type == 'fault':
            try:
                current_error = float(-np.log(current_error+0.999)) # if the type is fault, the error is negative
                print("Error transformed with log function")
            except ZeroDivisionError:
                float('inf')
        anomaly = current_error > self.novelty_threshold # check if the error is above the threshold
        
        self.err_dict['values'].append(current_error) # append the new error to the error array
        self.err_dict['timestamp'].append(self.snap['timestamp']) # append the new error to the timestamp array
        self.err_dict['assigned_cluster'].append(int(y)) # append the new error to the assigned_cluster array
        self.err_dict['anomaly'].append(anomaly) # append the new error to the assigned_cluster array
        print(f"Relative distance margin to the assigned cluster #{y}: {current_error}")
        if len(self.err_dict['values']) > self.__error_plot_size: # if the error array is bigger than the error queue size, remove the oldest error
            self.err_dict['values'] = self.err_dict['values'][1:] # remove the oldest error from the error array
            self.err_dict['timestamp'] = self.err_dict['timestamp'][1:] # remove the oldest error from the  timestamp array
            self.err_dict['assigned_cluster'] = self.err_dict['assigned_cluster'][1:] # remove the oldest error from the  assigned_cluster array
            self.err_dict['anomaly'] = self.err_dict['anomaly'][1:] # remove the oldest error from the  assigned_cluster array
        
        # save the error dictionary
        self.col_models.replace_one({'_id': f'Kmeans cluster {self.type} indicator'}, self.err_dict, upsert=True) # update the error dictionary
        n_anomaly = self.err_dict['anomaly'][-1-self.outlier_filter:] # get the last consecutive outlier_filter elements
        n_anomaly_mask = [True]*(self.outlier_filter+1) # all the allowed elements are True
        if n_anomaly == n_anomaly_mask: # if the number of anomalies is bigger than the outlier filter, move to quarantine
            print("alarm - NOVELTY DETECTED")
            match self.type:
                case 'novelty':
                    logging.warning(f"alarm - NOVELTY DETECTED in the sample with timestamp '{self.snap['timestamp']}'")
                case 'fault':
                    logging.warning(f"alarm - FAULT DETECTED in the sample with timestamp '{self.snap['timestamp']}'")
            return True  # return True if novelty/fault detected
        else:
            return False # return False if no novelty/fault detected
    def calculate_train_cluster_dist(self):
        ''' This method computes the maximum distance of each cluster in the train dataset '''
        self.labels_train_data  = self.kmeans.predict(self.trainMatrix) # predict the cluster of each sample in the train dataset
        self.cluster_distances  = self.kmeans.transform(self.trainMatrix) # gives the distance of each sample to each cluster
        self.train_cluster_dist=[] # maximum distance to eah cluster in the train dataset
        for cluster in range(0,self.num_clusters):
            self.train_cluster_dist.append(max(self.cluster_distances[ self.labels_train_data==cluster,cluster])) # get the maximum distance of the samples in the train dataset to tjis cluster
        
    def prepare_train_data(self):
        ''' This method prepares the training data for the MLA '''
        if not self.pack_train_data(): # if the healthy/faulty set is empty, nothing to update
            return False
        self.standardize_features()
        self.save_features_limits()
        self.save_StdScaler()
        return True
    def pack_train_data(self):
            """
            Packs the training data by appending healthy documents to the dataset.
            If the training set is empty, it is initialized with the oldest snapshot.
            """
            __train_data = self.col_train.find_one({'_id': 'training set'}) # find the training set
            if __train_data is None:  # if the training set is empty, initialize it with the oldest snapshot
                self.snap = self.col_features.find().sort('timestamp', pymongo.ASCENDING).limit(1)[0]  # get the oldest snapshot
                __id_to_remove = copy.deepcopy(self.snap['_id'])                  # copy the id of the snapshot to remove
                self.snap['_id']='training set'                                                      # rename it for initializing the training set
                self.col_train.insert_one(self.snap)                                                  # insert it in the training set   
                self.col_features.delete_one({'_id': __id_to_remove})                  # delete the snapshot from the features collection
                print("Training set initialized to '{self.col_train.full_name}' with '_id': 'training set'") 
            else:                   # append healty documents to the dataset
                cursor = self.col_features.find().sort('timestamp', pymongo.ASCENDING)  # get the oldest snapshot
                for self.snap in cursor:
                    if isinstance(__train_data['timestamp'],list):                     # if the training set is a list, pass
                        pass
                    else:                                                            # convert everityng to list
                        __train_data['timestamp'] = [__train_data['timestamp']]
                        for sensor in self.sensors:
                            for feature in __train_data[sensor].keys():
                                __train_data[sensor][feature] = [__train_data[sensor][feature]]          
                    __train_data['timestamp'].append( self.snap['timestamp'])                  # append the timestamp
                    for sensor in self.sensors:
                        for feature in __train_data[sensor].keys():
                            __train_data[sensor][feature].append(self. snap[sensor][feature])  # append the sensor data
                    self.col_features.delete_one({'_id': self.snap['_id']})                  # delete the snapshot from the features collection
                self.col_train.replace_one({'_id': 'training set'}, __train_data)         # replace the training set with the updated one 
                print(f"Training set updated to '{self.col_train.full_name}' with '_id': 'training set' ")
                return True # return True if the training set has been updated
    def __move_to_train(self,source:Collection):
        print(f"No data in the '{self.col_features.full_name}' collection, waiting for new data...")
        if typer.confirm(f"Do you want to move 'ALL' data  from '{source.full_name}' to '{self.col_features.full_name}'?",default=False):
            if source.count_documents({}) == 0:
                raise Exception("No data in the collection, cannot initialize the training set")
            self.moveCollection(source, self.col_features)
            self.snap = self.col_features.find().sort('timestamp',pymongo. ASCENDING).limit(1)[0]  # get the oldest snapshot
        else:
            print("No data in the collection, cannot initialize the training set...")
            #raise Exception("No data in the collection, cannot initialize the training set")
    def standardize_features(self):
        # now this method scales the data
        __train_data = self.col_train.find_one({'_id': 'training set'})             # get the training set
        if __train_data is None:
            raise Exception('Training set not initialized')
        _train_data_scaled = copy.deepcopy(__train_data)                                   # copy the training set
        _train_data_scaled['_id'] = 'training set scaled'                        # rename it
        self.features_minmax = copy.deepcopy(__train_data)                                   # copy the training set
        self.features_minmax['_id'] = 'training set MIN/MAX'                        # rename it
        
        print(id(__train_data), id(_train_data_scaled), id(self.features_minmax))
        
        # scale the features
        for sensor in self.sensors:
            self.StdScaler[sensor] = StandardScaler()
            __data = np.array(list(__train_data[sensor].values())) # the scaler wants the data in the form (n_samples, n_features)
            self.StdScaler[sensor].fit(__data.transpose()) # fit the scaler
            data_scaled = self.StdScaler[sensor].transform(__data.transpose()). transpose() # the scaler returns the data in the form (n_features, n_samples)
            data_scaled = data_scaled.tolist() # convert the data to list    
            for indx, feature in enumerate(_train_data_scaled[sensor].keys()):
                _train_data_scaled[sensor][feature] = data_scaled[indx] # the scaler returns the data in the form (n_features, n_samples)
                self.features_minmax[sensor][feature] = [float(np.min(data_scaled[indx])), float(np.max(data_scaled[indx]))]
        # save the scaled data
        self.col_train.delete_many({"_id": 'training set scaled'}) 
        self.col_train.insert_one(_train_data_scaled) 
        print(f"Training set scaled and saved into the collection '{self.col_train.full_name}' with '_id': 'training set scaled'")
    def save_features_limits(self):
        self.col_train.delete_many({"_id": 'training set MIN/MAX'}) 
        self.col_train.insert_one(self.features_minmax)
    def save_StdScaler(self):
        # save the scaler
        __pickled_data = pickle.dumps(self.StdScaler)
        try:
            self.col_train.insert_one({'_id': 'StandardScaler_pickled', 'data': __pickled_data})
        except:
            try:
                self.col_train.replace_one({'_id': 'StandardScaler_pickled'}, {'_id': 'StandardScaler_pickled', 'data': __pickled_data})
            except:
                raise Exception('Error saving the StandardScaler')
        print(f"StandardScaler saved as picled data into '{self.col_train.full_name}' with '_id': 'StandardScaler_pickled'")
    
    def save_KMeans(self):
        # save the scaler
        __pickled_data = pickle.dumps(self.kmeans)
        __id =f"KMeans_'{str(self.type)}'_pickled"
        try:
            self.col_models.delete_many({"_id": __id})
        except:
            pass # if the document is not found, pass
        try:
            self.col_models.insert_one({'_id': __id, 'data': __pickled_data})
        except:
            res = self.col_train.update_one({'_id': __id}, {'$set': {'_id': __id, 'data': __pickled_data}})
            print(f"Document with _id {__id} modified with result counter: {res.modified_count}")
            if res.modified_count == 0:
                raise Exception('Error saving the KMeans model')
        print(f"KMeans model saved as picled data into '{self.col_models.full_name}' with '_id': {__id}")
    
    def retrieve_KMeans(self):
            """
            Retrieves a KMeans model from the MongoDB collection specified by `self.col_models` and with an ID
            constructed from the model's `type` attribute. If the model is not found in the collection, an exception
            is raised. Otherwise, the model is loaded from the pickled data and stored in `self.kmeans`. The method
            also prints out the retrieved model's configuration.
            Raises:
                Exception: If the KMeans model is not found in the specified MongoDB collection.
            """
            __id =f"KMeans_'{str(self.type)}'_pickled"
            __retrieved_data: Collection | None = self.col_models.find_one({'_id': __id})
            if __retrieved_data is None:
                raise Exception('KMeans model not found in collection ' + self.col_models.full_name + ' with _id: ' + __id)
            else:
                self.kmeans:KMeans = pickle.loads(__retrieved_data['data'])
                print(f"KMeans retrieved from picled data @ {self.col_models.full_name}")
                print(f"with config:\n {self.kmeans.get_params()}")
                print(f"with n of features:\n {self.kmeans.n_features_in_}")
                pass
    
    def __retrieve_StdScaler(self):
        __retrieved_data: Collection | None = self.col_train.find_one({'_id': 'StandardScaler_pickled'})
        if __retrieved_data is None:
            raise Exception('Scaler not found in collection ' + self.col_train.full_name)
        else:
            self.StdScaler: Dict[str, StandardScaler] = pickle.loads(__retrieved_data['data'])
            print(f"StdScaler retrieved from picled data @ {self.col_train.full_name}")
    @staticmethod
    def retrieve_StdScaler(col: Collection):
        __retrieved_data: Collection | None = col.find_one({'_id': 'StandardScaler_pickled'})
        scaler : Dict[str, StandardScaler] | None
        if __retrieved_data is None:
            scaler = None
        else:
            scaler = pickle.loads(__retrieved_data['data'])
            return scaler
    def _append_features(self, col: Collection):
        ''' Append the features to the collection collection '''
        col.update_one({'_id': 'trainig set'}, {'$set': {}})
    
    def train(self):
        self.packFeaturesMatrix()       # pack the training features in a matrix
        self.__clusters_range = range(2,min(self.__max_clusters+1,self.trainMatrix.shape[0])) # range of clusters to evaluate
        self.evaluate_silhouette()
        self.evaluate_inertia()
        fig, axs=plt.subplots(1,2)
        fig.tight_layout()
        self.__plot_silhouette(axs[0])
        self.__plot_inertia(axs[1])
        print("Please decide the number of cluster for the training. The silhouette and inertia plots will be shown.")
        print("The silhouette should be maximized, while the inertia should be in a Pareto optimal point.")
        print("close the plot to continue...")
        plt.show()
        self.num_clusters=typer.prompt("Number of clusters", type=int)
        self.kmeans=KMeans(self.num_clusters,n_init='auto',max_iter= self.__max_iter) #reinitialize the kmeans
        self.kmeans.fit(self.trainMatrix)
        print(f"Kmeans trained with config:\n {self.kmeans.get_params()}")
        self.save_KMeans()
    
    def evaluate_silhouette(self):
        ''' This method evaluates the silhouette score for the training set '''
        self.__sil_score=[]
        for n_blobs in self.__clusters_range:
            __kmeans=KMeans(n_blobs,n_init='auto',max_iter= self.__max_iter)
            __y_pred_train=__kmeans.fit_predict(self.trainMatrix )
            self.__sil_score.append(silhouette_score(self.trainMatrix, __y_pred_train))
    
    def evaluate_inertia(self):
        self.__inertia=[]
        
        for n_blobs in self.__clusters_range:
            kmeans=KMeans(n_blobs,n_init='auto',max_iter=1000)
            kmeans.fit_predict(self.trainMatrix)
            self.__inertia.append(kmeans.inertia_)
    def packFeaturesMatrix(self):
        ''' This method packs the training features in a matrix'''
        __train_data = self.col_train.find_one({'_id': 'training set scaled'})             # get the training set
        if __train_data is None:
            raise Exception("'_id': 'training set scaled' not found in collection " + self.col_train.full_name)
        __features_values = []
        __features_names = []
        for sensor in self.sensors:
            for feature in __train_data[sensor].keys():
                __features_names.append(sensor + '_' + feature)
                __features_values.append(__train_data[sensor][feature])
        self.features_names, self.trainMatrix = (__features_names, np.array(__features_values).transpose())
        print("Features packed in a matrix: " + str(self.trainMatrix.shape))
    def __plot_silhouette(self, ax):
        ax.plot(self.__clusters_range,self.__sil_score)
        ax.set_ylabel('Silhouette')
        ax.set_xlabel('Num. of clusters')
    def __plot_inertia(self, ax):
        ax.plot(self.__clusters_range,self.__inertia)
        ax.set_ylabel('Inertia')
        ax.set_xlabel('Num. of clusters')
\end{minted}
\section{Exponential Regressor}
\label{sec:expreg}
\begin{minted}[linenos,tabsize=2,breaklines]{python}
def ExpRegressor(x,y):
    """
    Fits the function a*exp(b*x)+c to the given data points.
    Parameters:
    x (array-like): The x-coordinates of the data points.
    y (array-like): The y-coordinates of the data points.
    Returns:
    list: A list containing the fitted parameters [a,b,c].
    """
    # the original procedure fits a+b*e^(c*x): 
    x_f=np.array(x)
    y_f=np.array(y)
    S=np.array([0])
    for k in range(1,len(x_f)):
        S=np.append(S,S[-1]+0.5*(y_f[k]+y_f[k-1])*(x_f[k]-x_f[k-1]))
    A=np.linalg.norm(x_f-x_f[0])**2
    B=np.sum((x_f-x_f[0])*S)
    C=B
    D=np.linalg.norm(S)**2
    MAT1=np.matrix([[A,B],[C,D]])
    print(MAT1)
    A=np.sum((y_f-y_f[0])*(x_f-x_f[0]))
    B=np.sum((y_f-y_f[0])*S)
    MAT2=np.matrix([[A],[B]])
    print(MAT2)
    RES=MAT1**(-1)*MAT2
    print(RES)
    A1=RES[0]
    B1=RES[1]
    a1=-A1/B1
    c1=B1.copy()
    c2=B1.copy()
    theta=np.exp(c2*x_f)
    A=x_f.size
    B=np.sum(theta)
    C=B.copy()
    D=np.linalg.norm(theta)**2
    MAT1=np.matrix([[A,B],[C,D]])
    A=np.sum(y_f)
    B=np.sum(y_f*theta.reshape(-1,1))
    MAT2=np.matrix([[A],[B]])
    RES=MAT1**(-1)*MAT2
    A2=RES[0]
    B2=RES[1]
    return np.array([B2,c2,A2]).reshape(-1).tolist()
\end{minted}
\section{Database Manager}
\label{sec:dbmanager}
\begin{minted}[linenos,tabsize=2,breaklines]{python}
class DB_Manager:
    def __init__(self, configStr: str):
        self.configStr = configStr    #  path to config file (.yaml)
        try:
            with open(self.configStr,'r') as f:
                self.Config = yaml.safe_load(f)
                print(f'Loaded config file @ {self.configStr}')
        except:
            raise Exception(f'Error reading config file @ {self.configStr}')
        self.sensors: List[str] = list(self.Config['Database']['sensors'].keys()) # list of sensors4
        self.features = {}                                                              # initialize the features dict
        self.features["timestamp"] = None                                               # initialize the features dict with timestamp
        self.features.update({key: {} for key in self.sensors})                         # initialize the features dict with sensors                              
        self.client, self.db, self.col_back = mongoConnect( self.Config['Database']['db'], self.Config['Database']['collection']['back'], self.Config['Database']['URI'])
        _, _, self.col_raw = mongoConnect( self.Config['Database']['db'], self.Config['Database']['collection']['raw'], self.Config['Database']['URI'])
        _, _, self.col_unconsumed = mongoConnect( self.Config['Database']['db'], self.Config['Database']['collection']['unconsumed'], self.Config['Database']['URI'])
        _, _, self.col_healthy = mongoConnect( self.Config['Database']['db'], self.Config['Database']['collection']['healthy'], self.Config['Database']['URI'])
        _, _, self.col_healthy_train = mongoConnect( self.Config['Database']['db'], self.Config['Database']['collection']['healthy_train'], self.Config['Database']['URI'])
        _, _, self.col_quarantined = mongoConnect( self.Config['Database']['db'], self.Config['Database']['collection']['quarantined'], self.Config['Database']['URI'])
        _, _, self.col_faulty= mongoConnect( self.Config['Database']['db'], self.Config['Database']['collection']['faulty'], self.Config['Database']['URI'])
        _, _, self.col_faulty_train = mongoConnect( self.Config['Database']['db'], self.Config['Database']['collection']['faulty_train'], self.Config['Database']['URI'])
        _, _, self.col_models = mongoConnect( self.Config['Database']['db'], self.Config['Database']['collection']['models'], self.Config['Database']['URI'])
    
    def __repr__(self) -> str:
        return "Loded the configuration:\n" + str(self.Config)
    
    @staticmethod
    def eraseDB(configStr: str):
        '''
        delete the database specified in the config file
        '''
        try:
            Config = DB_Manager.loadConfig(configStr)
        except:
            raise Exception(f'Error reading config file @ {configStr}')
        client  = MongoClient(Config['Database']['URI'])                                    # connect to MongoBD
        if Config['Database']['db'] in client.list_database_names():
            client.drop_database(Config['Database']['db'])                                  # delete database
            print(f'Deleted database \'{Config["Database"]["db"]}\' @ \'{Config["Database"]["URI"]}\'')
        else:
            print(f'Tried to delete database, but Database \'{Config["Database"]["db"]}\' not found @ \'{Config["Database"]["URI"]}\'')
        client.close()
    @staticmethod
    def createEmptyDB(configStr: str):
        '''
        create an empty database with the collections specified in the config file.
        '''
        try:
            Config = DB_Manager.loadConfig(configStr)
        except:
            raise Exception(f'Error reading config file @ {configStr}')
        client  = MongoClient(Config['Database']['URI'])                                    # connect to MongoBD
        if Config['Database']['db'] in client.list_database_names():
            raise Exception(f'Database \'{Config["Database"]["db"]}\' already exists @ \'{Config["Database"]["URI"]}\'')
        else:
            db = client[Config['Database']['db']]                                           # create database
            print(f'Created empty database \'{Config["Database"]["db"]}\' @ \'{Config["Database"]["URI"]}\'')
            for cols in Config['Database']['collection'].values():
                db.create_collection(cols)                                                  # create empty collections
                print(f'Created empty collection \'{cols}\' @ \'{Config["Database"]["db"]}\'')
        client.close() 
    @staticmethod
    def loadConfig(configStr: str):
        '''
        load the configuration file
        '''
        try:
            with open(configStr,'r') as f:
                Config = yaml.safe_load(f)
        except:
            raise Exception(f'Error reading config file @ {configStr}')
        return Config                                                                     # close connection
    
    def moveCollection(self,source_collection: Collection,destination_collection: Collection, keep_source: bool = False):
        '''
        move all the documents from one collection to another   
        '''
        # Query all documents from the source collection
        documents_to_move = source_collection.find({})
        # Iterate through the documents and insert them into the destination collection
        for document in documents_to_move:
            destination_collection.insert_one(document)
        print(f'All documents copied from {source_collection.full_name} and inserted into {destination_collection.full_name}')  
        if not keep_source:
            source_collection.delete_many({}) # delete all the documents from the source collection
            print(f'All documents deleted from {source_collection.full_name}') 
    
    def _find_snap(self,_id:str, col: Collection):
        ''' find the snapshot with the specified timestamp in the specified collection '''
        self.snap = col.find({'_id': _id})[0]            #pick the right snapshot
    def _read_features(self, col: Collection, order = pymongo.ASCENDING):
        ''' Read the data from the collection - put data in self.snap
            return True if data are available, False otherwise '''
        try:
            self.snap    = col.find().sort('timestamp',order).limit(1)[0]     # oldest/newest record - sort gives a cursor, the [0] is the dict
            print(f"Imported snapshot with timestamp {self.snap['timestamp']} from {col}")
            return True    
        except IndexError:
            print(f"No data in collection {col.full_name}, waiting for new data...")
            return False
        
    def _write_snap(self, col: Collection):
        ''' Write the data to the collection '''
        try:
            col.insert_one(self.snap)
            print(f"Inserted snapshot with timestamp {self.snap['timestamp']} into {col.full_name}")
        except:
            print(f"Error inserting snapshot with timestamp {self.snap['timestamp']} into {col.full_name}")
        
    def _replace_snap(self, col: Collection):
        ''' Replace the data in the collection '''
        col.replace_one({'_id': self.snap['_id']},self.snap)
        print(f"Replaced snapshot with timestamp {self.snap['timestamp']} in {col.full_name}")
    
    def _delete_snap(self, col: Collection):
        ''' delete the data from the collection '''
        col.delete_one({'_id': self.snap['_id']})
        print(f"Deleted snapshot with timestamp {self.snap['timestamp']} from {col.full_name}")
    
\end{minted}