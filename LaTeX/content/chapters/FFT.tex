\chapter{Fourier Transform}
\label{app:FFT}
This appendix aims to provide a brief and non-exhaustive introduction to the Fourier Transform. The various types of Fourier Transform are key tools in a vast range of fields in engineering, physics and mathematics. In computer science, the Fourier Transform is used in signal processing, image processing, data compression, and many other applications.

\section{Continuous Fourier Transform}
\label{sec:ContinuousFourierTransform}
Most of the modern signal processing techniques find their roots in the Fourier Transform, which is a mathematical tool that allows the decomposition of a non-periodic signal into its frequency components. For periodic signals, The transform is a train of impulses, whose amplitudes are linked to the Fourier series coefficients.

The Continuous Fourier Transform (\gls{cft}) of a signal $x(t)$ is defined as:
$$
X(f) = \int_{-\infty}^{\infty} x(t) e^{-j2\pi ft} dt, \qquad \forall f \in \mathbb{R}
$$
where $X(f)$ is the Fourier Transform of $x(t)$, $f$ is the frequency variable and $j$ is the imaginary unit. The quantity $2\pi f$ is the angular frequency.

\section{Discrete Fourier Transform}
\label{sec:DiscreteFourierTransform}
The Discrete Fourier Transform (\gls{dft}) is a sampled version of the Continuous Fourier Transform. 
From an engineering perspective, the DFT is of particular interest because most of the signals are sampled in time and therefore, the \gls{dft} is the most commonly used form of the Fourier Transform.

The \gls{dft} is used to transform a sequence of $N$ complex numbers $x(n)$ into another sequence of $N$ complex numbers $X(k)$. The DFT is defined as:
\begin{equation}
\label{eq:DFT}
X(k) = \sum_{n=0}^{N-1} x(n) e^{-j2\pi kn/N}, \qquad \forall k \in \{0, 1, \ldots, N-1\}
\end{equation}
where $X(k)$ is the \gls{dft} of $x(n)$, $k$ is the frequency index and $N$ is the number of samples in the sequence.

The output spectrum given by the \gls{dft} is also discrete and finite, with the same number of samples as the input signal. 
The sampling frequency of the signal $x(n)$ determines the upper frequency limit of the \gls{dft} output, as the faster the sampling rate, the higher the frequencies that can be represented. The length of the signal $x(n)$ determines the frequency resolution of the \gls{dft} output, as the longer the signal, the finer the frequency resolution is.

\paragraph{Computational complexity}
Examining the definition of the \gls{dft} in Equation \ref{eq:DFT}, we can notice that it requires $N$ complex multiplications and $N(N)$ complex additions. Therefore, the computational complexity of the \gls{dft} is $\mathcal{O}(N^2)$, which does not scale well for large values of $N$.

\section{Fast Fourier Transform}
\label{sec:FastFourierTransform}

To overcome the computational complexity of the \gls{dft}, the Fast Fourier Transform (\gls{fft}) algorithm was developed in the 1960s by Cooley and Tukey \cite{cooley1965algorithm}. Choosing a data length $N$ that is a power of 2, the \gls{fft} algorithm reduces the computational complexity of the \gls{dft} from $\mathcal{O}(N^2)$ to $\mathcal{O}(N \log_2 N)$. This becomes particularly important for large values of $N$, where the \gls{fft} algorithm is significantly faster than the \gls{dft}.

In the classic algorithm, a single multiplication happens many times. The general idea of this fast algorithm is to exploit the periodicity of the complex exponentials in the \gls{dft} to divide the computation into smaller sub-problems and reuse the already computed multiplication. 

Breaking the exponential into its sine and cosine components, it becomes intuitive that the same result appears multiple times in the computation. The \gls{fft} algorithm takes advantage of this redundancy to reduce the number of operations, recursively dividing the problem into half-sized subproblems up to the point that the subproblems have only one sample.
