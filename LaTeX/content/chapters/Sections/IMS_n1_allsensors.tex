\section{\gls{ims} dataset No.1 - All sensors}
In the previous section, an extensive test of the framework has been performed on a single signal from the dataset (Bearing 3 x). So the warning given by the \gls{mla} was detecting a problem in a specific component of the maintained system. Let's now test a configuration that takes into account all the signals of the dataset, so all eight signals from the four bearings are used for feature extraction. This configuration should be able to detect a generic novel behavior of the system or, better, a situation in which the system is abnormal as a whole (the signals may be normal but the combination of them may be abnormal).  In this case, the configuration file has been set to use all the time-domain and all the frequency-domain features, and the \gls{mla} has been trained with the same procedure as before, with the first 600 snapshots of the dataset. 


\begin{figure}
    \centering
    \includegraphics{images/IMS/Novelty_01_500samples_allsensors.pdf}
    \caption{Novelty detection on the \gls{ims} dataset No.1 using all the sensors}
    \label{fig:IMS_n1_allsensors}
\end{figure}

In \autoref{fig:IMS_n1_allsensors}, the evolution of the novelty metric over time is shown. Ignoring an outlier appeared at the end of 2003-11-19, the \gls{nd} event triggered at 2003-11-20 23:44, 5~days before the end of the data acquisition. After the event, the novelty metric stays consistently over the threshold.

Let's compare now this result on the classic approach of just using the RMS of the signals. The maximum value of the RMS among all the bearings is shown in \autoref{fig:IMS_n1_allsensors_RMS}. It is clear that the \gls{ml} approach is much more robust, as there is no threshold that would have both minimizet the false positives and triggered a \gls{nd} event sufficiently in advance.

\begin{figure}
    \centering
    \includegraphics{images/IMS/Novelty_01_RMS_allsensors.pdf}
    \caption{Maximum value of RMS vibration among all bearings}
    \label{fig:IMS_n1_allsensors_RMS}
\end{figure}