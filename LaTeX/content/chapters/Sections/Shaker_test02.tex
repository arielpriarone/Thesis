\section{Experiments on a laboratory shaker - Test 2}
\label{sec:shaker_test02}
In the previous section, the first test on the shaker was presented. The test has shown the capability of the framework to detect unknown harmonics. A second test was done to evaluate the capability to detect time-domain variations and the effect of reducing the frequency resolution. 

\subsection{Training and evaluating}
This new configuration has been set to use only 4 frequency-domain \gls{glo:feature}s and the same 3 time-domain \gls{glo:feature}s of the previous test, for a total of 7 \gls{glo:feature}s. The signal used for training and testing is resumed in \autoref{tab:shaker_param_02}. The set is composed of the same signal at different amplitudes used for training and testing, plus another signal with different frequency content but the same amplitude as a training signal used for testing. 

The training has been carried out in the same way as the previous test, the training of the K-means model has been done with 4 \gls{glo:clust}s, and loaded on the microcontroller. 

\begin{table}
    \centering
    \caption{Parameters of the second shaker test.}
    \label{tab:shaker_param_02}
    \begin{tabular}{cccccccc} 
    \toprule
    \multicolumn{5}{c}{\textbf{Harmonic frequency} {[}Hz]} & \multirow{2}{*}{\textbf{Amplitude }{[}mV$_{pp}$]} & \multicolumn{2}{c}{\textbf{ No. of \gls{glo:snap}s}} \\
    10 & 30 & 60 & 70 & 100 &  & Train & Test \\ 
    \hline
    - & 0.1 & - & 1.0 & 1.0 & 580 & 100 & 10 \\
    - & 0.1 & - & 1.0 & 1.0 & 1000 & 100 & 10 \\
    - & 0.1 & - & 1.0 & 1.0 & 1980 & 100 & 10 \\
    - & 0.1 & - & 1.0 & 1.0 & 1540 & 100 & 10 \\
    - & 0.1 & - & 1.0 & 1.0 & 2000 & - & 20 \\
    - & 0.1 & - & 1.0 & 1.0 & 0 & - & 10 \\
    - & 0.1 & - & 1.0 & 1.0 & 800 & - & 10 \\
    - & 0.1 & - & 1.0 & 1.0 & 200 & - & 10 \\
    - & 0.1 & - & 1.0 & 1.0 & 1220 & - & 10 \\
    1.0 & 1.0 & 0.1 & - & - & 1540 & - & 10 \\
    \bottomrule
    \end{tabular}
    \end{table}



\subsection{Results}
\begin{figure}
    \centering
    \includegraphics{Images/shaker/Test02.pdf}
    \caption{Novelty detection result}
    \label{fig:shaker_results02}
\end{figure}
The result of the novelty detection is shown in \autoref{fig:shaker_results02}. The first 4 lines have been correctly identified as normal, as they were in fact a repetition of the training signals.
Then the purple and cyan line in the figure is the same training signal, but 20 mV higher in amplitude \gls{wrt} the training signal. The novelty metric overshoots the threshold in 5 samples out of 20. An increase of 2\% in amplitude generates the \gls{nd} event 25\% of the times can be observed with this signal. 

The brown, grey and light-green lines are the same signal, but with a bigger difference in amplitude \gls{wrt} the training signal. All the \gls{glo:snap}s of these signals correctly generated a novelty metric above the threshold. The blue line is the signal with a different frequency content, and it has been correctly identified as a novelty event, this is the confirmation that even with just 4 frequency bins, the wavelet decomposition is still generating \gls{glo:feature}s that are informative.

The pink line is the test signal with an amplitude of 800mV. It's evident that the novelty metric is below the threshold, and the signal has been classified as normal even if it is not in the training dataset. Let's investigate how this happened. The first consideration is that the 800mV amplitude is quite tight to both the 1000mV and 580mV signals used for training. Moreover, in this case, the total number of \gls{glo:feature}s is just 7. This allows plotting all the \gls{glo:feature}s against each other, to see why the \gls{nd} event has not been detected. In \autoref{fig:shaker_conf_matrix} the scatter plot of the \gls{glo:feature}s is shown. It's evident that, in this environment, even performing the standardization of the \gls{glo:feature}s, the \gls{glo:clust}s are still very elongated, resembling almost a line. To fit an elongated \gls{glo:clust} in a hypersphere, it is inevitable that in some sections, the hypersphere will not closely surround the \gls{glo:clust}, leaving \quoted{space} for false negative results. Another problem is that the k-means algorithm tends to split long \gls{glo:clust}s. In the figure, the red dots are the false negative results, and the grey shades are the hypersphere projection on the considered \gls{glo:feature}s plane. The black dots are the training data. The effect of the elongated \gls{glo:clust}s is particularly evident in the plot of the \quoted{Feature 3} against \quoted{Feature 2}, where the red dots are in between two \gls{glo:clust}s, that are modelled as one. On the other hand, looking at the plot of \quoted{Feature 1} against \quoted{Feature 4}, a very long \gls{glo:clust} has been split in two. This is an example of exploiting the limitations of the k-means algorithm anticipated in \autoref{sec:kmeans_limits}. 
For completness, in \autoref{fig:shaker_conf_matrix}, also the true positive results are shown, as magenta dots.


\begin{figure}
    \centering
    \includegraphics{Images/shaker/ConfusionMatrix.pdf}
    \caption{False Negative and True Positive results. On the diagonal, there is a histogram of the \gls{glo:feature} values. The off-diagonal plots are the scatter plots of the \gls{glo:feature}s. The shades are the projection of the \gls{glo:clust}s on the considered plane. (Red: False Negative, Magenta: True Positive, Black: training data)}
    \label{fig:shaker_conf_matrix}
\end{figure}
\clearpage

\subsection{Possible improvements}
The environment of this test is very challenging for the k-means algorithm. As discussed in \autoref{ch:Unsupervised}, there are algorithms that are not affected by the \gls{glo:clust}s' shapes. The candidate algorithms that may perform better in this situation are the \gls{lof}, the \gls{iforest} and \gls{dbscan}. Future work could be to implement these algorithms in the \gls{glo:edge} framework, despite being more demanding in computational power and memory, and test them in this environment.

As proof of concept, the \gls{lof} implementation in \texttt{\gls{glo:python}} has been used to perform \gls{nd} on the same dataset used in this section in \gls{glo:edge}. The results are reported in \autoref{label:lof_results}. The \gls{lof} algorithm has been able to correctly identify all the \gls{nd} events, even the signals with just 20mV variation from the training dataset, and the 800mV signal that was problematic for the K-means. The \gls{lof}, however, generated a false positive on the 580mV signal. This false positive may be avoided by increasing the threshold, but this would also increase the false negative rate. 

\begin{figure}
    \centering
    \includegraphics{Images/shaker/Test02_LOF.pdf}
    \caption{\gls{lof} novelty detection result}
    \label{fig:lof_results}
\end{figure}