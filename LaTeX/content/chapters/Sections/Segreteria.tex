The fourth industrial revolution is characterized by the integration of Artificial Intelligence and the Internet of Things paradigm into factories. Nowadays, more than a decade after this industrial revolution began, the maintenance approach remained unchanged in most industrial applications. The primary factor impeding the advancement of the maintenance approach is the significant expense associated with implementing Condition-Based or Predictive maintenance strategies, coupled with a lack of knowledge about the modelling or behaviour of a failing system.

In most facilities, maintenance continues to be performed according to a predefined schedule. An optimization of this approach involves intervening in the system only when necessary, which requires the knowledge of when a system is malfunctioning. Fault Detection and Anomaly Detection enable the triggering of an event when a known fault occurs or when a new, unfamiliar behaviour emerges in the maintained system. Predictive Maintenance aims to predict the Remaining Useful Life of a system. A quite novel frontier is the direct implementation of Predictive Maintenance within the maintained system, a concept known as Edge Computing.

In this work, to ease the implementation into different systems, a modular and general-purpose framework is proposed. To overcome the lack of models, it is developed following an Unsupervised Machine Learning approach. The Machine Learning core of the framework is based on the features extracted from the data gathered from sensors. The proposed framework operates in real-time, continuously assessing the health of the system. It has been developed to be executed and tested on a PC using various Unsupervised Machine Learning algorithms, implemented in the Python language. The algorithms that appeared to maximize the performance-resources ratio were deployed on a microcontroller using the C language. The proposed solution includes all the tools necessary in the data pipeline. With this structure, the framework can be easily set up on a machine and extended to an arbitrary configuration of sensors and features. The PC implementation underwent testing using various Unsupervised algorithms on publicly available datasets, while the edge implementation was tested through laboratory experiments.

Both the test on datasets and the experimental results showed that the proposed framework is able to detect anomalies and give an estimate of the future degradation evolution of the system. 