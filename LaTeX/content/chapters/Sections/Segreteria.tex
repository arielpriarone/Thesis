Nowadays, the fourth industrial revolution is taking place, and it is characterized by the integration of Artificial Intelligence and the Internet of Things paradigm into factories. At the same time, in many industrial applications, the maintenance approach remained unchanged for decades. The main reason that holds back the evolution of the maintenance approach is the high costs of implementation of Condition-Based or Preventive maintenance strategies and a lack of knowledge about the modelling or behaviour of a failing system.

A framework that can be used for detecting a novelty behaviour (ND), a known fault (FD), and estimating the Remaining Useful Life (RUL) of the maintained system is proposed. To overcome the lack of models an Unsupervised Machine Learning (UML) approach is used and, on the other hand, to overcome the difficulties of implementation on different systems a modular and general-purpose framework has been developed.

A quite new frontier in the field of Predictive Maintenance is the implementation of the Artificial Intelligence application directly in the maintained system, this approach is called Edge Computing.

In this work, the framework has been developed to be run and tested on a PC using various Unsupervised Machine Learning algorithms, Using the Python language. The algorithms that appeared to maximize the performance-resources ratio have been then implemented in a microcontroller, using the C language. The edge implementation has been tested with laboratory experiments.

The proposed solution relies on the application of Machine Learning to features extracted by the time-domain signal by the framework itself. With this structure, the framework can be easily set up on a machine, read signals from sensors, extract features, and then train the models. The framework then switches to evaluation mode to perform ND, FD, and RUL predictions. After a Nd or FD event, the model can be refined performing a retraining using only the data that generated the event.

The considered features are settable for each sensor. The implementation of the feature extraction manages time-domain features (Mean, RMS, Standard Deviation, P2P, Skewness, Kurtosis) and frequency-domain features (The power of coefficients of a Wavelet Packet Decomposition of any depth, or the FFT of the signal). The framework is designed to be easily extended to additional features and sensors.

The PC implementation is based on Software Agents that act autonomously on a common database. The following algorithms have been implemented to perform ND, FD, and RUL predictions: K-means, DBSCAN, Gaussian Mixture Models, One-Class Support Vector Machines, Isolation Forest and Local Outlier Factor. The K-means model was then chosen for the Edge implementation.

All the cited algorithms' performances have been compared on a dataset of bearings vibration published online by the Center for Intelligent Maintenance Systems (IMS) of the University of Cincinnati. The Edge implementation of the framework has been tested in laboratory experiments, using an accelerometer to measure the vibrations generated by an active shaker and, on a second set of tests, by a linear actuator.

The last refinement of the framework has been the implementation of additional feature scaling after the standardization, using a Random Forest model to refine the UML model. This mitigates the effect that noisy features have on the performance of the algorithms. This approach has been tested on the laboratory data.