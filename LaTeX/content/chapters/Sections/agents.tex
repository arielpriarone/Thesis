\section{Software Agents}
\label{sec:agents}

\begin{figure}
    \centering
    \includegraphics[width=\textwidth]{images/Framework/Framework_structure.pdf}
    \caption{Framework logical structure}
    \label{fig:Framework_structure}
\end{figure}

In the previous sections, the software agents were mentioned as the main actors in the framework. This section will provide a more detailed description of the software agents, their role and their interaction with the environment and the data, following the flow from the hardware through the time-series, the feature domain to the \gls{ml} algorithms. The reference layout is the one in \autoref{fig:Framework_structure}. 

Software \gls{glo:agent}s are autonomous programs that perform a specific task in a cycle. In the proposed \texttt{\gls{glo:python}} implementation, the agents are classes that are instanced and run in a loop.

\subsection{Field Agent}
\label{subsec:FieldAgent}
\begin{figure}
    \centering
    \includegraphics[scale=1]{images/Framework/Field_Agent_flowchart.pdf}
    \caption{Field Agent flowchart}
    \label{fig:Field_Agent_flowchart}
\end{figure}

The \gls{glo:fieldagent} is the interface between the hardware and the software. It is responsible for the acquisition of the data from the sensors and the communication with the database. Since some features are related to the spectrum of the data, a precise and fixed sampling frequency is needed. Hence, the \gls{fieldAg} must incorporate a synchronization with the \gls{adc}. It stores data in the \emph{raw} collection and the \emph{backup} collection. In \autoref{fig:Field_Agent_flowchart}, the flow of operations is shown as a flowchart, emphasizing the importance of the synchronization with the \gls{adc}. This software agent has not been implemented in \texttt{\gls{glo:python}}, because the experimental validation of this work, as it will be described in \autoref{sec:Validation}, has been performed directly on the \gls{glo:edge} platform. During the tests on the publicly available datasets, an abstract version of the \gls{fieldAg} has been used, that reads the data from the \gls{csv} files.


\subsection{Feature Agent (\gls{fa})}
\label{subsec:FeatureAgent}
\begin{figure}
    \centering
    \includegraphics[width=\textwidth]{images/Framework/FA_flowchart.pdf}
    \caption{Feature Agent flowchart}
    \label{fig:FA_flowchart}
\end{figure}

The \gls{fa} is responsible for the feature extraction from the raw data. It reads the data from the \emph{raw} collection, extracts the features and stores them in the \emph{unconsumed} collection. The flow of operation is shown in \autoref{fig:FA_flowchart}. The \gls{fa} is implemented in \texttt{\gls{glo:python}} and it is a class that has been designed to be easily expandable and configurable. The methods implemented in the \gls{fa} class are shown in \autoref{tab:FA_methods}.


\begin{longtable}{p{0.4\textwidth}p{0.5\textwidth}}
    \caption{\gls{fa} class implemented methods\label{tab:FA_methods}}\\ 
    \toprule
    \textbf{Method} & \textbf{Description} \endfirsthead
    \hline
    readFromRaw & reads a \gls{glo:snap} from the raw collection and stores it in the instance self \\
    extractFeatures & extract all the features from the current \gls{glo:snap}s, for all the sensors \\
    extractTimeFeautures & extract~mean,~rms, P2P, std, skewness and kurtosis, based on the config file for the specified sensor \\
    extractFreqFeautures & extract the wavelet coefficients for the specified sensor, up to the configured depth \\
    deleteFromraw & delete current snap record from the \emph{raw} collection \\
    writeToUnconsumed & write the extracted features to the \emph{unconsumed} collection \\
    initialize\_barPlotFeatures & initializes the bar plot of the features that is shown to the user \\
    barPlotFeatures & updates the bar plot with new features \\
    run & perform the agent operations in a loop. idle until new data are available in \emph{raw} collection \\
    \bottomrule
\end{longtable}
    

\subsection{Machine Learning Agent (\gls{mla})}
\label{subsec:MLA}
The \gls{mla} is responsible for the training and the evaluation of the \gls{ml} models. It reads the data from the \emph{unconsumed} collection, evaluates the \gls{glo:snap} and stores the result in the \emph{models} collection, it also constantly updates the information about the novelty or fault metric to the user. The flow of operation is shown in \autoref{fig:MLA_structure}. The methods implemented in the \gls{mla} class are shown in \autoref{tab:MLA_methods}.

This agent is designed to be configured as a \gls{nd} or \gls{fd} agent with just one hyperparameter. If it is instanced for \gls{nd}, it uses the healthy collection as a training dataset, if it is instanced for \gls{fd} it uses the faulty collection. The metric used to evaluate the \gls{glo:snap}s is the novelty metric for \gls{nd} and the fault metric for \gls{fd}. According to the procedure defined in \autoref{alg:eval_new_snapshot}.

\begin{figure}[htbp]
    \centering
    \includegraphics[width=\textwidth]{images/Framework/MLA.pdf}
    \caption{Machine Learning Agent flowchart. When it is instanced for \gls{nd}, the \gls{mla} uses the healthy collection as a training dataset, when it is instanced for \gls{fd} it uses the faulty collection.}
    \label{fig:MLA_structure}
\end{figure}


\begin{longtable}{p{0.4\textwidth}p{0.5\textwidth}}
    \caption{\gls{mla} class implemented methods\label{tab:MLA_methods}}\\ 
    \toprule
    \textbf{Method} & \textbf{Description} \endfirsthead 
    \hline
    run & run the \gls{mla} according to its current state \\
    evaluate & evaluate the current \gls{glo:snap} based on the novelty or the fault metric, according to the type of instance \\
    predict & fits the novelty metric with a degradation curve to predict the future evolution~ \\
    mark\_snap\_evaluated & set the evaluated flag to true for the current \gls{glo:snap} \\
    delete\_evaluated\_snap & remove the evaluated \gls{glo:snap}s from the \emph{unconsumed} collection \\
    scale\_features & scales the features of the current \gls{glo:snap} according to the standard scaler used during the training procedure \\
    evaluate\_error & compute the novelty or fault metric for the current \gls{glo:snap}, according to the type of instance \\
    calculate\_train\_cluster\_dist & compute the radiuses of the \gls{glo:clust}s during the training procedure \\
    prepare\_train\_data & performs the preprocessing of the data before training the model \\
    pack\_train\_data & pack the training \gls{glo:snap} in a matrix \\
    \_\_move\_to\_train & move an entire collection of \gls{glo:snap}s to the training collection \\
    standardize\_features & make all the features in the training matrix have zero mean and unitary variance \\
    save\_features\_limits & save the unscaled bounds of the training features \\
    save\_StdScaler & store the standard scaler instance in \gls{glo:pickle} into the database \\
    retrieve\_StdScaler & restore the standard scaler instance in \gls{glo:pickle} from the database \\
    save\_KMeans & store the model instance in \gls{glo:pickle}  \\
    retrieve\_KMeans & restore the model instance in \gls{glo:pickle}  \\
    \_append\_features & append the current features in a document \\
    train & performs the training of the \gls{glo:clust}ing models \\
    evaluate\_silhouette & compute the silhouette score of the training set \gls{glo:snap}s \\
    \_\_plot\_silhouette & plots the silhouette score against the number of \gls{glo:clust}s \\
    evaluate\_inertia & compute the inertia score of the training set \gls{glo:snap}s \\
    \_\_plot\_inertia & plots the inertia score against the number of \gls{glo:clust}s \\
    packFeaturesMatrix & format all the training features as a matrix \\
    retrain & perform a new training of the models \\
    \bottomrule
    \end{longtable}
    
\newpage
\subsection{Configuration of the framework}
All the configurations described in this chapter are stored in the \texttt{config.yaml} file. This file is read by the agents at the beginning of their execution. The configuration file is divided into sections by topic: database, models etc. The \autoref{tab:yaml} shows the structure of the configuration file.


\begin{longtable}{>{\hspace{0pt}}m{0.26\linewidth}>{\hspace{0pt}}m{0.113\linewidth}>{\hspace{0pt}}m{0.569\linewidth}}
  \caption{Structure of the framework configuration file.\label{tab:yaml}}\\ 
  \toprule
  \rowcolor[rgb]{0.929,0.929,0.923}\textbf{Field} & \textbf{Type} & \textbf{Description} \endfirsthead 
  \hline
  Database & structure & Database configuration \\
  \rowcolor[rgb]{0.929,0.929,0.923}$\dots$/URI & string & MongoDB URI \\
  $\dots$/DB & string & Database name \\
  \rowcolor[rgb]{0.929,0.929,0.923}$\dots$/Collection & structure & Collections configuration \\
  $\dots$/$\dots$/Backup & string & Backup collection name \\
  \rowcolor[rgb]{0.929,0.929,0.923}$\dots$/$\dots$/Raw & string & Raw data collection name \\
  $\dots$/$\dots$/Unconsumed & string & Unconsumed data collection name \\
  \rowcolor[rgb]{0.929,0.929,0.923}$\dots$/$\dots$/Healthy & string & Healthy data collection name \\
  $\dots$/$\dots$/Healthy\_train & string & Healthy data collection packed for training (some healthy data are not used if not novelty) \\
  \rowcolor[rgb]{0.929,0.929,0.923}$\dots$/$\dots$/Quarantined & string & Quarantined data collection name \\
  $\dots$/$\dots$/Faulty & string & Faulty data collection name \\
  \rowcolor[rgb]{0.929,0.929,0.923}$\dots$/$\dots$/Faulty\_train & string & Faulty data collection packed for~training (some faulty data are not used if not novelty) \\
  $\dots$/$\dots$/Models & string & Models collection name \\
  \rowcolor[rgb]{0.929,0.929,0.923}Sensors & structure & Sensors configuration \\
  $\dots$/Sensor 1/Name & string & Sensor 1 name \\
  \rowcolor[rgb]{0.929,0.929,0.923}$\dots$/$\dots$/Features & structure & Features configuration of this sensor \\
  $\dots$/$\dots$/$\dots$/Wavelet & bool & Enable or disable wavelet decomposition for the considered sensor \\
  \rowcolor[rgb]{0.929,0.929,0.923}$\dots$/$\dots$/$\dots$/Mean & bool & Enable or disable mean feature for the considered sensor \\
  $\dots$/$\dots$/$\dots$/\gls{rms} & bool & Enable or disable \gls{rms} feature for the considered sensor \\
  \rowcolor[rgb]{0.929,0.929,0.923}$\dots$/$\dots$/$\dots$/P2P & bool & Enable or disable P2P feature for the considered sensor \\
  $\dots$/$\dots$/$\dots$/Std & bool & Enable or disable Standard deviation feature for the considered sensor \\
  \rowcolor[rgb]{0.929,0.929,0.923}$\dots$/$\dots$/$\dots$/Skew & bool & Enable or disable skewness feature for the considered sensor \\
  $\dots$/$\dots$/$\dots$/Kurt & bool & Enable or disable kurtosis feature for the considered sensor \\
  \rowcolor[rgb]{0.929,0.929,0.923}$\dots$/$\dots$ & $\dots$ & $\dots$ \\
  $\dots$/Sensor n/Name & string & Sensor $n$ name \\
  \rowcolor[rgb]{0.929,0.929,0.923}$\dots$/Sensor\_n/Features & structure & Features configuration of this sensor \\
  $\dots$/Sensor n/$\dots$ & $\dots$ & $\dots$ \\
  \rowcolor[rgb]{0.929,0.929,0.923}Wavelet & structure & Wavelet Packet Decomposition configuration \\
  $\dots$/Type & string & Type of wavelet to be used (ex. db10, Moore, etc.) \\
  \rowcolor[rgb]{0.929,0.929,0.923}$\dots$/Mode & string & Mode of decomposition (ex. Symmetric) \\
  $\dots$/Level & int & Depth of the decomposition tree (No. of features~$=2^\text{Level}$) \\
  \rowcolor[rgb]{0.929,0.929,0.923}Model & structure & Models configuration \\
  $\dots$/Max \gls{glo:clust}s & int & Maximum number of \gls{glo:clust}s to attempt during \gls{glo:clust}ing \\
  \rowcolor[rgb]{0.929,0.929,0.923}$\dots$/Max iterations & int & Maximum iterations of the \gls{uml} algorithm \\
  $\dots$/queue & int & Number of \gls{rul} predictions to keep in memory \\
  \rowcolor[rgb]{0.929,0.929,0.923}$\dots$/Plot size & int & Number of Novelty Metric values to be kept in memory for plotting \\
  Novelty-Fault & structure & Configuration of the novelty or fault detection \\
  \rowcolor[rgb]{0.929,0.929,0.923}$\dots$/Threshold & float & Novelty - Fault detection threshold \\
  $\dots$/\gls{rul} Threshold & float & Novelty - Fault threshold for \gls{rul} predictions \\
  \rowcolor[rgb]{0.929,0.929,0.923}$\dots$/Fit points & int & Number of samples used for fitting the prediction curve \\
  $\dots$/Outlier filter & int & Number of consecutive outliers to filter \\
  \rowcolor[rgb]{0.929,0.929,0.923}$\dots$/Regressor & string & Type of curve to fit (exp or scipy to select the closed form solution or the iterative solution) \\
  Log & string & Path where to store the logs of the framework \\
  \bottomrule
  \end{longtable}
  

\subsection{Command Line Interface (\gls{cli})}
\label{subsec:CLI}

To ease the interaction with the user, a \gls{cli} has been implemented. It relies on the \texttt{click} and \texttt{typer} libraries for \texttt{\gls{glo:python}}. The \gls{cli} allows the user to instance the agents, to configure the framework, to monitor the agents and to interact with the database. All the commands are provided with a help message that can be accessed by typing \texttt{{-}{-}help} after the command, as shown in \autoref{fig:cli} for the command \texttt{run-machine-learning-agent}.
The commands implemented in the \gls{cli} are shown in \autoref{tab:CLI_commands}.

\begin{figure}[h!]
  \centering
  \includegraphics[width=\textwidth]{images/Framework/cli.png}
  \caption{Command Line Interface help message}
  \label{fig:cli}
\end{figure}

\begin{longtable}{p{0.4\textwidth}p{0.5\textwidth}}
    \caption{\gls{cli} implemented commands\label{tab:CLI_commands}}\\ 
    \toprule
    \textbf{Command} & \textbf{Description} \endfirsthead 
    \hline
    copy-collection            &Move all the documents from one collection to another\\
    create-empty-db            &Create an empty database in MongoDB. The database should not exist already. It is configured according with "config.yaml" file.\\
    ims-converter              &Transfer the data from the gls{ims} textual files into the MongoDB database in a suitable way.\\
    fault-indicator            &This function plots the fault metric.\\
    novelty-indicator          &This function plots the novelty metric.\\
    move-collection            &Move all the documents from one collection to another\\
    plot-features              &Plot the features of the last \gls{glo:snap} in the UNCONSUMED collection\\
    run-feature-agent          &Run the Feature Agent - takes the last \gls{glo:snap} from RAW collection, extracts features and writes them to UNCONSUMED collection\\
    run-machine-learning-agent &Run the Machine Learning Agent \\
    \bottomrule
    \end{longtable}