\chapter{Conclusion and future work}
\label{ch:Conclusion}

The proposed \gls{glo:frmwrk} has demonstrated the capability of detecting anomalies in generic time domain data. The proposed modular structure is a key for enabling the \gls{glo:frmwrk} to be considered a prototype that can be applied to a wide range of applications. The unsupervised learning approach further eases and accelerates the adaptation of this solution into different domains. 

The deployment of the solution also for \gls{glo:edge} enables the \gls{glo:frmwrk} to expand the range of applicability beyond classical industrial environments, toward non-standard applications. The intrinsic cybersecurity of edge devices is a nice byproduct that can be exploited whenever the \gls{glo:frmwrk} is deployed in a critical infrastructure.

In future work, there is significant potential for further testing and refinement of the already developed \gls{glo:frmwrk}. One path to explore involves testing the edge implementation on a degrading system rather than relying on simulated degradation through environmental parameter changes. This approach could provide invaluable insights into the robustness and adaptability of the \gls{glo:frmwrk} in critical scenarios. Additionally, deploying the remaining algorithms within the edge \gls{glo:frmwrk} to address the limitations of K-means \gls{glo:clust}ing presents an exciting opportunity. Since the repertoire of algorithms shown promising results on real word datasets, deploying all of them also in the edge version of the \gls{glo:frmwrk} allows to offer a more comprehensive solution to complex problems in \gls{glo:edge} environments.

The \gls{glo:frmwrk} could also be enhanced from the user experience perspective by developing a Graphical User Interface (\gls{gui}) that allows to interact with the \gls{glo:frmwrk} in a more intuitive way. The already developed \gls{cli} could remain a secondary option for advanced users, or be used in the background by the \gls{gui} to execute the \gls{glo:frmwrk}'s functionalities. 