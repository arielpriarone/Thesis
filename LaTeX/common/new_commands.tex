
% how to change Contents to Table of Contents
\addto\captionsenglish{% Replace "english" with the language you use
  \renewcommand{\contentsname}%
    {Table of Contents}%
}

% to change the name of Abbreviations to Acronyms
% not needed if use use entry types and define those
% \renewcommand{\abbreviationsname}{Acronyms}

% to allow line comments in algorithms
\algnewcommand{\LineComment}[1]{\State \(\triangleright\) #1}

% to declare abs and norm
\DeclarePairedDelimiter\abs{\lvert}{\rvert}%
\DeclarePairedDelimiter\norm{\lVert}{\rVert}%

% Swap the definition of \abs* and \norm*, so that \abs
% and \norm resizes the size of the brackets, and the 
% starred version does not.
\makeatletter
\let\oldabs\abs
\def\abs{\@ifstar{\oldabs}{\oldabs*}}
%
\let\oldnorm\norm
\def\norm{\@ifstar{\oldnorm}{\oldnorm*}}
\makeatother

%% *** ARIEL PRIARONE ***
\let\dummy\autoref % This redefine autoref as dummy
\def\autoref#1{\textbf{\dummy{#1}}} % this is like \autoref but in bold
\newcommand{\algorithmautorefname}{algorithm} % to make autoref work with algorithms
\addto\extrasenglish{\def\figureautorefname{figure}}%
\addto\extrasenglish{\def\chapterautorefname{chapter}}%
\addto\extrasenglish{\def\sectionautorefname{section}}%
\addto\extrasenglish{\def\subsectionautorefname{subsection}}%
\addto\extrasenglish{\def\subsubsectionautorefname{subsubsection}}%
\addto\extrasenglish{\def\paragraphautorefname{paragraph}}%
\addto\extrasenglish{\def\tableautorefname{table}}%
\addto\extrasenglish{\def\equationautorefname{equation}}%

\newcommand{\vect}[1]{\bm{#1}}  % for vectors
\newcommand{\quoted}[1]{``#1''} % for quotes
\newcommand{\argmin}[1]{\mathrm{arg}\,\underset{#1}{\mathrm{min}}}
\newcommand{\argmax}[1]{\mathrm{arg}\,\underset{#1}{\mathrm{max}}}
\newcommand{\citepage}[2]{\cite[p.~#2]{#1}}

\DeclareMathOperator*{\E}{\mathbb{E}}

\colorlet{punct}{red!60!black}
\definecolor{delim}{RGB}{20,105,176}
\colorlet{numb}{magenta}
\definecolor{eclipseStrings}{RGB}{42,0.0,255}
\definecolor{eclipseKeywords}{RGB}{0,100,0}
\lstdefinelanguage{json}{
    basicstyle=\scriptsize\ttfamily,
    commentstyle=\color{eclipseStrings}, % style of comment
    stringstyle=\color{eclipseKeywords}, % style of strings
    numbers=left,
    numberstyle=\scriptsize,
    stepnumber=1,
    numbersep=8pt,
    showstringspaces=false,
    breaklines=true,
    frame=lines,
    string=[s]{"}{"},
    comment=[l]{\#},
    morecomment=[l]{:"},
    literate=
        *{0}{{{\color{numb}0}}}{1}
         {1}{{{\color{numb}1}}}{1}
         {2}{{{\color{numb}2}}}{1}
         {3}{{{\color{numb}3}}}{1}
         {4}{{{\color{numb}4}}}{1}
         {5}{{{\color{numb}5}}}{1}
         {6}{{{\color{numb}6}}}{1}
         {7}{{{\color{numb}7}}}{1}
         {8}{{{\color{numb}8}}}{1}
         {9}{{{\color{numb}9}}}{1}
}

\def\checkmark{\tikz\fill[scale=0.4](0,.35) -- (.25,0) -- (1,.7) -- (.25,.15) -- cycle;} % to have a checkmark
\addto{\captionsenglish}{\renewcommand\summaryname{Abstract}} % A.P. 2024 - to change the name of the summary in the ToC
\newcommand{\customClearDoublePage}{\clearpage{\thispagestyle{empty}\mbox{}\cleardoublepage}} % to have a blank page and start on a fresh right page

\newcommand{\mask}[1]{          #1} % to mask text
\newcommand{\maskk}[1]{           } % to mask text in introduction
\newcommand{\maskabstract}[1]{  #1} % to mask abstract
\newcommand{\maskglossaries}[1]{#1} % to mask glossaries
\newcommand{\todo}[0]{            } % still something to do here
