\section{INTRODUCTION}
\label{sec:introduction}

Predictive Maintenance (PM) and Novelty Detection (ND) are important topics in modern industrial engineering, focused on proactively identifying equipment failures before they affect system functionality. Embracing these practices is crucial for reducing equipment downtime and optimizing maintenance efforts. PM aims to quantify and forecast the state of degradation of a system. A quite novel frontier is the direct implementation of PM algorithms within the maintained device, using the principles of Edge Computing.

\subsection{Motivation}
Despite the Fourth Industrial Revolution, the maintenance approach remained unchanged in many industrial applications. The primary factor impeding the advancement of the maintenance approach is the significant expense associated with implementing PM strategies, coupled with a lack of knowledge about the modelling or behaviour of a failing system.

According to a recent survey by the U.S. Department of Commerce, establishments focusing on preventing equipment failures experience 3.3 times less downtime compared to those only fixing issues after they occur.

\subsection{Objective}
The goal of this project is to design, develop and test a \emph{degradation} based framework
that performs ND using one or several Unsupervised Machine Learning (UML) algorithms. 
The structure of the framework is thought to be modular and general-purpose to ease the implementation into different systems. The framework has been developed following an unsupervised approach to overcome the common lack of physical models and labelled data of the maintained device. It has been deployed for both PC and Edge Computing.