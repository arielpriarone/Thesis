\section{VALIDATION}
\label{sec:validation}

\subsection{On bearings vibration datasets}

The PC implementation of the framework has been thoroughly tested on a publicly available dataset collected by the Center for Intelligent Maintenance Systems (IMS).
%\footnote{\url{https://www.nasa.gov/intelligent-systems-division/discovery-and-systems-health/pcoe/pcoe-data-set-repository/}}
The dataset contains time series of bearings vibrations collected during three separate \quoted{run to failure} experiences. 

\begin{figure}
    \includegraphics[width=\linewidth]{images/ND_IMS.pdf}
    \caption{Results of ND on the IMS dataset. When the Novelty Metric crosses the threshold, a warning is issued. The malfunction happens at the end of the dataset.}
    \label{fig:ND_IMS}
\end{figure}
\begin{table}
    \centering
    \caption{Comparison of the results for the test n$^\circ$1 of IMS dataset.}
    \label{tab:ims01_comparision}
    \begin{tabular}{lcrr} 
    \toprule
    \textbf{Algorithm} & \textbf{ND event} & \textbf{LT }{[}min] & \textbf{LT }{[}days] \\ 
    \hline
    K-means & 2003-11-16 07:46 & \textbf{13913} & \textbf{9.6} \\
    DBSCAN & 2003-11-22 15:06 & 4833 & 3.3\\
    GMM & 2003-11-22 03:47 & 5513 & 3.8\\
    BGMM & 2003-11-22 03:45 & 5514 & 3.8\\
    $\nu$-SVM & 2003-11-22 14:56 & 4844 &3.3\\
    IF & 2003-11-16 10:08 & 13771 & 9.6\\
    LOF & 2003-11-16 07:48 & 13912 & 9.6\\
    {P2P} without any ML & 2003-11-22 16:06 & 4774 & 3.3\\
    \bottomrule
    \end{tabular}
\end{table}
The ND capability of the framework has been tested using all the cited UML models. The Lead Time (LT) elapsed between the ND event and the actual fault is used to compare the models. The results are compared in table~\ref{tab:ims01_comparision}. The evolution of the NM, using a K-means model on a signal of the test N$^\circ$1, is shown in Fig.~\ref{fig:ND_IMS}.

\begin{figure}
    \includegraphics[width=\linewidth]{images/RUL_IMS.pdf}
    \caption{Results of Remaining Useful Life predictions on the IMS dataset. The Vertical lines indicate the time when the predictions are performed. The RUL is the time remaining before the fitted curve crosses the threshold.}
    \label{fig:RUL_IMS}
\end{figure}
To perform PM, the framework has to estimate the Remaining Useful Life (RUL) of the component. The RUL is predicted by a curve fitted to the data. The type of curve to fit is configurable. In this work $y = a \cdot e^{b \cdot x} + c$ has been used. The results of the RUL predictions are shown in Fig.~\ref{fig:RUL_IMS}.

\begin{figure}
    \includegraphics[width=\linewidth]{images/FD_IMS.pdf}
    \caption{Results of FD on the IMS dataset. The malfunction is detected when the Fault Metric crosses the threshold. The RUL is the time remaining before the fitted curve crosses the threshold.}
    \label{fig:FD_IMS}
\end{figure}
Since the second and third tests of the IMS dataset share the same type of fault, the framework has been trained to perform FD with the faulty data of the second test and then it has been evaluated on the third test. Analogously to the ND, when the FM crosses a certain threshold, a warning is issued and RUL predictions are performed. The results of the FD are shown in Fig.~\ref{fig:FD_IMS}.

\subsection{Laboratory tests}

\begin{figure}
    \includegraphics[width=\linewidth]{images/EmbeddedStructure.pdf}
    \caption{Structure of the Edge implementation. The microcontroller autonomously evaluates the status of the maintained system. Only during the training phase, the PC implementation computes the model for the Edge.}
    \label{fig:embedded}
\end{figure}

K-means, IF and LOF are the three models that performed the best on the dataset. Notably, the K-means model has advantages such as minimal parameter storage requirements and low computational costs for calculating the NM. For these reasons, the K-means model has been chosen for the Edge implementation.

In the first phase, the framework gathers the data and extracts the features. The UML model is trained on a PC and then included in the Edge. After, the microcontroller autonomously evaluates the status of the maintained system. The structure of the Edge implementation is shown in Fig.~\ref{fig:embedded}. 

\begin{figure}
    \includegraphics[width=\linewidth]{images/Test02_LOF.pdf}
    \caption{Results of a laboratory test on the shaker. The first five lines are evaluations of known vibrations, correctly flagged as normal samples. The remaining lines are the successful detections of the novelty of new vibrations.}
    \label{fig:shaker}
\end{figure}

Extensive tests have been performed on a laboratory shaker, simulating the vibrations generated by a generic mechanical system. The tests have been carried out to evaluate the sensitivity to variations in both amplitude and frequency content. The results of a ND test are shown in Fig.~\ref{fig:shaker}.

A second series of tests has been performed mounting the accelerometer on a linear actuator. The axis of the accelerometer has been aligned with the direction of the movement of the actuator to sense the actuated acceleration, rather than the vibrations. A set of predefined movement profiles has been used for training, while another set for testing. This series of tests exploited a high number of non-significant features. This is because the WPD gave high resolution over a wide range of frequency content, but the actuator excited only a few, and very low, frequencies. The remaining features were almost all noise, and the standardization procedure had the side effect of amplifying them to the same level as the significant ones. An effort has been made to fine-tune the model to reduce the impact of the noise (see Fig.~\ref{fig:linear}, models 1 to 4). The best result, however, has been obtained by removing the less informative features and using a reduced feature space. The benefit of this approach is evident in Fig.~\ref{fig:linear}.


\begin{figure}
    \includegraphics[width=\linewidth]{images/linear.pdf}
    \caption{Results of laboratory test on linear actuator acceleration profiles. The dimensionality of the feature space has been reduced to train Model~5, which performs a clear and sharp detection of the alternating pattern between known and unknown movement profiles.}
    \label{fig:linear}
\end{figure}