Nowadays, the fourth industrial revolution is taking place, and it is characterized by the integration of the Internet of Things paradigm. At the same time, in most industrial applications, the maintenance approach remained unchanged for decades. The main reason that holds back the evolution of the maintenance approach is the high costs of implementation of Condition-Based or Preventive maintenance strategies and a lack of knowledge about the modeling or behavior of a failing system.

A framework that can be used for detecting a novelty behavior (ND), a known fault (FD), and estimating the Remaining Useful Life (RUL) of the maintained system is proposed. To overcome the lack of models an Unsupervised Machine Learning approach is used, on the other hand, to overcome the difficulties of implementation on different systems a modular and general-purpose approach has been developed.

A quite new frontier in the field of Predictive Maintenance is the implementation of the Artificial Intelligence application directly in the maintained system, this approach is called Edge Computing.

In this work, the framework has been developed to be run and tested on a PC using various Unsupervised Machine Learning algorithms, Using the Python language. The algorithms that appeared to maximize the performance-resources ratio have been then implemented in a microcontroller, using the C language. The edge implementation has been tested with laboratory experiments.

The PC implementation is based on Software Agents that act autonomously on a common database. The following algorithms have been implemented to perform ND, FD, and RUL predictions: K-means, DBSCAN, Gaussian Mixture Models, One-Class Support Vector Machines, Isolation Forest and Local Outlier Factor. The K-means model was then chosen for the Edge implementation.

All the cited algorithms have been compared on a dataset of bearings vibration published by the Intelligent Maintenance System of the University of Cinci